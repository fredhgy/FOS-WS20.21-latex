%last updated 01.11.19 by GuangyuHe

\documentclass[twoside]{article}
%\documentclass[UTF8]{ctexart}
\setlength{\oddsidemargin}{0.25 in}
\setlength{\evensidemargin}{-0.25 in}
\setlength{\topmargin}{-0.6 in}
\setlength{\textwidth}{6.5 in}
\setlength{\textheight}{8.5 in}
\setlength{\headsep}{0.75 in}
\setlength{\parindent}{0 in}
\setlength{\parskip}{0.1 in}

%
% ADD PACKAGES here:
%

\usepackage[UTF8]{ctex}

\usepackage{amsmath,amsfonts,graphicx}
\usepackage{esint}
\usepackage{multicol}
%\usepackage{MnSymbol}
\usepackage{mathtools}
\usepackage[bottom]{footmisc}
\usepackage[pdfpagelabels,hyperindex,hyperfigures,breaklinks,colorlinks,linkcolor=black,citecolor=black,urlcolor=black]{hyperref}
\usepackage{geometry}
\usepackage{marginnote}
\usepackage{color}
%
% The following commands set up the lecnum (lecture number)
% counter and make various numbering schemes work relative
% to the lecture number.
%

\newcounter{lecnum}
\renewcommand{\thepage}{\thelecnum-\arabic{page}}
%\renewcommand{\thesection}{\thelecnum.\arabic{section}}
%\renewcommand{\theequation}{\thelecnum.\arabic{equation}}
%\renewcommand{\thefigure}{\thelecnum.\arabic{figure}}
%\renewcommand{\thetable}{\thelecnum.\arabic{table}}

%
% The following macro is used to generate the header.
%
\newcommand{\lecture}[6]{
   \pagestyle{myheadings}
   \thispagestyle{plain}
   \newpage
   \setcounter{lecnum}{#1}
   \setcounter{page}{1}
   \noindent
   \begin{center}
   \framebox{
      \vbox{\vspace{2mm}
    \hbox to 6.28in { {\bf #5
	\hfill #6} }
       \vspace{4mm}
       \hbox to 6.28in { {\Large \hfill Lecture #1: #2  \hfill} }
       \vspace{2mm}
       \hbox to 6.28in { {\it Lecturer: #3 \hfill Scribes: #4} }
      \vspace{2mm}}
   }
   \end{center}
   \markboth{Lecture #1: #2}{Lecture #1: #2}

   %{\bf Note}: {\it LaTeX template courtesy of UC Berkeley EECS dept.}

   %{\bf Disclaimer}: {\it These notes have not been subjected to the
   %usual scrutiny reserved for formal publications.  They may be distributed
  % outside this class only with the permission of the Instructor.}
   \vspace*{4mm}
}

\newcommand{\exercise}[6]{
   \pagestyle{myheadings}
   \thispagestyle{plain}
   \newpage
   \setcounter{lecnum}{#1}
   \setcounter{page}{1}
   \noindent
   \begin{center}
   \framebox{
      \vbox{\vspace{2mm}
    \hbox to 6.28in { {\bf #5
	\hfill #6} }
       \vspace{4mm}
       \hbox to 6.28in { {\Large \hfill Exercise #1: #2  \hfill} }
       \vspace{2mm}
       \hbox to 6.28in { {\it Lecturer: #3 \hfill Scribes: #4} }
      \vspace{2mm}}
   }
   \end{center}
   \markboth{Exercise #1: #2}{Exercise #1: #2}
   
   %{\bf Note}: {\it LaTeX template courtesy of UC Berkeley EECS dept.}

   %{\bf Disclaimer}: {\it These notes have not been subjected to the
   %usual scrutiny reserved for formal publications.  They may be distributed
  % outside this class only with the permission of the Instructor.}
  \vspace*{4mm}
}

\newcommand{\review}[4]{
   \pagestyle{myheadings}
   \thispagestyle{plain}
   \newpage
   %\setcounter{lecnum}{#1}
   \setcounter{page}{1}
   \noindent
   \begin{center}
   \framebox{
      \vbox{\vspace{2mm}
    \hbox to 6.28in { {\bf #3
	\hfill #4} }
       \vspace{4mm}
       \hbox to 6.28in { {\Large \hfill Review \hfill} }
       \vspace{2mm}
       \hbox to 6.28in { {\it Lecturer: #1 \hfill Scribes: #2} }
      \vspace{2mm}}
   }
   \end{center}
   \markboth{Review}{Review}
   
   %{\bf Note}: {\it LaTeX template courtesy of UC Berkeley EECS dept.}

   %{\bf Disclaimer}: {\it These notes have not been subjected to the
   %usual scrutiny reserved for formal publications.  They may be distributed
  % outside this class only with the permission of the Instructor.}
  \vspace*{4mm}
}

\newcommand{\answer}[6]{
   \pagestyle{myheadings}
   \thispagestyle{plain}
   \newpage
   \setcounter{lecnum}{#1}
   \setcounter{page}{1}
   \noindent
   \begin{center}
   \framebox{
      \vbox{\vspace{2mm}
    \hbox to 6.28in { {\bf #5
	\hfill #6} }
       \vspace{4mm}
       \hbox to 6.28in { {\Large \hfill Answer Sheet #1: #2  \hfill} }
       \vspace{2mm}
       \hbox to 6.28in { {\hfill #3 , #4} }
      \vspace{2mm}}
   }
   \end{center}
   \markboth{Answer Sheet #1: #2}{Answer Sheet #1: #2}
   
   %{\bf Note}: {\it LaTeX template courtesy of UC Berkeley EECS dept.}

   %{\bf Disclaimer}: {\it These notes have not been subjected to the
   %usual scrutiny reserved for formal publications.  They may be distributed
  % outside this class only with the permission of the Instructor.}
  \vspace*{4mm}
}


%
% Convention for citations is authors' initials followed by the year.
% For example, to cite a paper by Leighton and Maggs you would type
% \cite{LM89}, and to cite a paper by Strassen you would type \cite{S69}.
% (To avoid bibliography problems, for now we redefine the \cite command.)
% Also commands that create a suitable format for the reference list.
\renewcommand{\cite}[1]{[#1]}
\def\beginrefs{\begin{list}%
        {[\arabic{equation}]}{\usecounter{equation}
         \setlength{\leftmargin}{2.0truecm}\setlength{\labelsep}{0.4truecm}%
         \setlength{\labelwidth}{1.6truecm}}}
\def\endrefs{\end{list}}
\def\bibentry#1{\item[\hbox{[#1]}]}

%Use this command for a figure; it puts a figure in wherever you want it.
%usage: \fig{NUMBER}{CAPTION}{picture scale}{picture name}
\newcommand{\fig}[4]{
    \begin{center}
        \includegraphics[scale=#1]{fig//#2}
    \end{center}
    \begin{center}
        Figure #3 #4
    \end{center}


    %\begin{center}
    %    \includegraphics[scale=#3]{#4}
    %\end{center}
    %\vspace{2mm}
    %\begin{center}            
    %    Figure \thelecnum.#1:~#2
    %\end{center}
}
% Use these for theorems, lemmas, proofs, etc.
%\newtheorem{theorem}{Theorem}[lecnum]
%\newtheorem{lemma}[theorem]{Lemma}
%\newtheorem{proposition}[theorem]{Proposition}
%\newtheorem{claim}[theorem]{Claim}
%\newtheorem{corollary}[theorem]{Corollary}
%\newtheorem{definition}[theorem]{Definition}
%\newenvironment{proof}{{\bf Proof:}}{\hfill\rule{2mm}{2mm}}

% **** IF YOU WANT TO DEFINE ADDITIONAL MACROS FOR YOURSELF, PUT THEM HERE:

\newcommand\E{\mathbb{E}}

\newcommand{\nn}{
    \nonumber \\
}%不带公式编号换行

\newcommand{\ketn}[1]{
    | #1 \rangle
}%右矢

\newcommand{\bran}[1]{
    \langle #1 |
}%左矢

\newcommand{\braoket}[1]{
    \langle #1 | #1 \rangle
}

\newcommand{\bratket}[2]{
    \langle #1 | #2 \rangle
}

\newcommand{\inpro}[2]{
    \langle #1 , #2 \rangle
}%内积

\newcommand{\rightarrowtxt}[1]{
    \stackrel{\text{#1}}{\longrightarrow}
}%箭头上带文字

\newcommand{\ahat}{
    \hat{a}
}%下降算符

\newcommand{\adag}{
    \hat{a}^\dagger
}%上升算符

\newcommand{\osrt}{
    \frac{1}{\sqrt{2}}
}%根二分之一

\newcommand{\abs}[1]{
    \left| #1 \right|
}%绝对值

\newcommand{\sumint}{
    \mathclap{\displaystyle\int}\mathclap{\textstyle\sum}
}%叠加积分

\newcommand{\longline}{
    \rule{\textwidth}{0.5mm}
} %页面分割线

\newcommand{\explain}[1]{
    \footnote{{\it #1}}
}%脚注+斜体说明

\newcommand{\sidemark}[1]{
    \reversemarginpar
    \marginnote{\textsl{side remark:} \\ #1 }[3cm]
}%侧边注

\newcommand{\highlightquestion}[1]{
    {\color{red} #1 }
}%红色高亮问题

\newcommand{\highlight}[1]{
    {\color{blue} #1 }
}%蓝色高亮重点

\newcommand{\sbeq}{
    \overset{!}{=}
}%should be equal

\newcommand{\bigkuohao}[2]{
    \left\{
    \begin{aligned}
   #1 \\
    #2
    \end{aligned}
    \right.
}

\newcommand{\txtoneq}[1]{
    \stackrel{\text{#1}}{=}
}%等号上带文字

\newcommand{\hate}{
    \hat{e}
}

\newcommand{\veca}{
    \vec{A}
}

\newcommand{\vecb}{
    \vec{B}
}

\newcommand{\vecd}{
    \vec{D}
}

\newcommand{\vece}{
    \vec{E}
}

\newcommand{\vech}{
    \vec{H}
}

\newcommand{\vecj}{
    \vec{J}
}

\newcommand{\vecjj}{
    \vec{j}
}

\newcommand{\veck}{
    \vec{k}
}

\newcommand{\vecp}{
    \vec{P}
}

\newcommand{\vecpp}{
    \vec{p}
}

\newcommand{\vecs}{
    \vec{S}
}

\newcommand{\vecr}{
    \vec{r}
}

\newcommand{\vecnabla}{
    \vec{\nabla}
}

\newcommand{\partialt}{
    \frac{\partial}{\partial t}
}

\newcommand{\partialtt}{
    \frac{\partial^2}{\partial t^2}
}

\newcommand{\labtag}[1]{
    \label{tag: #1}\tag{#1}
}







%command for physics uses


%\begin{document}
    % Some general latex examples and examples making use of the
% macros follow.  
%**** IN GENERAL, BE BRIEF. LONG SCRIBE NOTES, NO MATTER HOW WELL WRITTEN,
%**** ARE NEVER READ BY ANYBODY.

%\section{Some theorems and stuff} % Don't be this informal in your notes!

%\begin{lemma} %引理
%This is the first lemma of the lecture.
%\end{lemma}

%\begin{proof} %证明
%The proof is by induction on $\ldots$.
%For fun, we throw in a figure.
%%%NOTE USAGE !
%\fig{1}{1in}{A Fun Figure} %插入图片说明

%This is the end of the proof, which is marked with a little box.
%\end{proof}

%\subsection{A few items of note}

%Here is an itemized list:
%\begin{itemize}
%\item this is the first item;
%\item this is the second item.
%\end{itemize}

%Here is an enumerated list:
%\begin{enumerate}
%\item this is the first item;
%\item this is the second item.
%\end{enumerate}

%Here is an exercise:

%{\bf Exercise:}  Show that ${\rm P}\ne{\rm NP}$.

%Here is how to define things in the proper mathematical style.
%Let $f_k$ be the $AND-OR$ function, defined by

%\[ f_k(x_1, x_2, \ldots, x_{2^k}) = \left\{ \begin{array}{ll}

%	x_1 & \mbox{if $k = 0$;} \\

%	AND(f_{k-1}(x_1, \ldots, x_{2^{k-1}}),
%	   f_{k-1}(x_{2^{k-1} + 1}, \ldots, x_{2^k}))
%	 & \mbox{if $k$ is even;} \\

%	OR(f_{k-1}(x_1, \ldots, x_{2^{k-1}}),
%	   f_{k-1}(x_{2^{k-1} + 1}, \ldots, x_{2^k}))	
%	& \mbox{otherwise.} 
%	\end{array}
%	\right. \]

%\begin{theorem}
%This is the first theorem.
%\end{theorem}

%\begin{proof}
%This is the proof of the first theorem. We show how to write pseudo-code now.
%*** USE PSEUDO-CODE ONLY IF IT IS CLEARER THAN AN ENGLISH DESCRIPTION

%Consider a comparison between $x$ and~$y$:
%\begin{tabbing}
%\hspace*{.25in} \= \hspace*{.25in} \= \hspace*{.25in} \= \hspace*{.25in} \= \hspace*{.25in} \=\kill
%\>{\bf if} $x$ or $y$ or both are in $S$ {\bf then } \\
%\>\> answer accordingly \\
%\>{\bf else} \\
%\>\>    Make the element with the larger score (say $x$) win the comparison \\
%\>\> {\bf if} $F(x) + F(y) < \frac{n}{t-1}$ {\bf then} \\%
%\>\>\> $F(x) \leftarrow F(x) + F(y)$ \\
%\>\>\> $F(y) \leftarrow 0$ \\
%\>\> {\bf else}  \\
%\>\>\> $S \leftarrow S \cup \{ x \} $ \\
%\>\>\> $r \leftarrow r+1$ \\
%\>\> {\bf endif} \\
%\>{\bf endif} 
%\end{tabbing}

%This concludes the proof.
%\end{proof}


%\section{Next topic}

%Here is a citation, just for fun~\cite{CW87}.

%\section*{References}
%\beginrefs
%\bibentry{CW87}{\sc D.~Coppersmith} and {\sc S.~Winograd}, 
%``Matrix multiplication via arithmetic progressions,''
%{\it Proceedings of the 19th ACM Symposium on Theory of Computing},
%1987, pp.~1--6.
%\endrefs

% **** THIS ENDS THE EXAMPLES. DON'T DELETE THE FOLLOWING LINE:
%\end{document}
\begin{document}
\lecture{1}{20.11.11}{Prof. Arno Rauschenbeutel}{Guangyu He}{Fundamentals of Optical Sciences:experimental part}{WS 2020/21}

\section{Ray Optics}
    {\bf Description of all optical phenomena $\leftrightarrow$ quantum optics(quantum electrodynamics)}

    But: Depending on the problem, various approximations can be (and should be!) applied.
    \begin{itemize}
        \item light is classical coupled vector waves for electromagnetic field $\leftrightarrow$ electromagnetic optics
        \item light is classical scalar wave $\leftrightarrow$ wave optics
        \item light is rays, which propagate following geometric rules $\leftrightarrow$ ray optics
    \end{itemize}

    We have:

    ray optics $\subset $ wave optics $\subset $ electrodynamics optics $\subset $ quantum optics,

    where, historically, the corresponding theories have been developed is about this order.
    
    \subsection{Postulates of Ray Optics}
        \begin{itemize}
            \item light propagates in the form of rays
            \item optic media are characterized by their refractive index, $n\geq 1$, where 
            \begin{align}
                n= \frac{c_0}{c} \labtag{1.1}
            \end{align}
            and $c_0$: speed of light in vacuum, $c$: speed of light in medium
        \end{itemize}

        propagation time over distance d:
        \begin{align}
            t = \frac{d}{c} = \frac{nd}{c_0} \nonumber
        \end{align}

        where $nd$: optical path length, $d$: geometrical path length

        \begin{itemize}
            \item inhomogeneous medium $\leftrightarrow$ $n$depends on position $\vecr$
        \end{itemize}
        Optic path length in this case:
        \begin{align}
            \int_A^B  n\left(\vecr\right)ds \nonumber
        \end{align}

        \fig{0.5}{ex_fosvl1_1.png}{1}{}

        \begin{itemize}
            \item Fermat's principle: light rays propagate from A to B such that the optic path length takes an external value:
            \begin{align}
                \delta  \int_A^B  n\left(\vecr\right)ds = 0 \labtag{1.2}
            \end{align}
        \end{itemize}
        
        In most cases, the optical path length will be minimized, thus,

        \underline{light rays propagate from A to B along a path that takes the least time.}

        N.B. : Fermat's principle allows one to derive, ??? others, the laws of reflection and refraction.

        \underline{law of reflection:}\\
        reflected ray lies in the place of incidence angle of incidence = ??? angle
        \fig{0.5}{ex_fosvl1_2.png}{2}{law of reflection}

        \underline{law of refraction:}\\
        refracted ray lies in the plane of incidence, $\theta_1$ and $\theta_2$ connected by Snell's laws:
        \begin{align}
            n_1\sin\theta_1 = n_2\sin\theta_2 \labtag{1.3}
        \end{align}
        \fig{0.5}{ex_fosvl1_3.png}{3}{law of reflection}

    \subsection{Simple optical components}
        \underline{Plane mirror}

        \fig{0.5}{ex_fosvl1_4.png}{4}{ All rays from $P_1$ are reflected ??? that they seem to issue from image point $P_2$}

        \underline{Pasalolic mirror}

        Collects all rays that propagate parallel to the optical axis in focal point F$_1$ \\
        $\bar{PF} =: f$ is focal length
        \fig{0.5}{ex_fosvl1_5.png}{5}{}

        \underline{Elliptical mirror}
        
        \fig{0.5}{ex_fosvl1_6.png}{6}{ All rays that issues from focal point $P_1$ are reflected into 2nd focal point $P_2$}

        {\bf Fermat's principle $\Rightarrow$ optical path lengths of all rays are identical($\Leftrightarrow$ definition of the ellipse)}

        \underline{Sepherical mirrors}
        
        Remark: These are easier to fabricate from a technical point of view. For paraxial rays(near optical axis \& small angle w.r.t optical axis), 
        they have approximate the source imaging properties as parabolic \& elliptical mirrors.
        
        Image formation for paraxial rays:
        \fig{0.5}{ex_fosvl1_7}{7}{Convention: $R < 0$ for concave surface, and $R > 0$ for convex surface        }
        \begin{align}
            \frac{1}{z_1} + \frac{1}{z_2} \simeq \frac{2}{-R} = \frac{1}{f} \labtag{1.4}
        \end{align}

        \underline{gaussian mirror equation}\\
        (a.k.a. mirror and lens equations)

        {\bf For rays II to optical axis, we have $z_1 =\infty$ and, thus, $z_2 = (-R)/2$. Hence, $f=(-R)/2$ is the focal length.}

        \underline{Plane interface}

        \fig{0.5}{ex_fosvl1_8}{8}{ray is reflected away from the interface$\rightarrow \theta_1 < \theta_2$ }

        \fig{0.5}{ex_fosvl1_9}{9}{ray is refracted towards the interface $\rightarrow \theta_2 > \theta_1$}

        from a certain angle ???, i.e., for $\theta_1 \leq \theta_c$, where $\theta_c$ is ?? critical angle, we have $\theta_2 \leq 90^°$ and the refracted 
        ray no longer exists.

        We have
        \begin{align}
            n_1 \sin\theta_c = n_2 \sin90^° = n_2 \nn
            \Rightarrow \theta_c = \arcsin(n_2 / n_1) \labtag{1.5}
        \end{align}

        \fig{0.5}{ex_fosvl1_10}{10}{Relation between $\theta_1 \& theta_2$}

        {\bf For $\theta_1 > \theta_2$, all light is reflected at the interface $\rightarrow$ total internal reflection}

        Total internal reflection is frequently employed in order to efficiently reflect or guide light:
        
        \fig{0.5}{ex_fosvl1_11}{11}{TIR for $n_1 > \sqrt{2}n_2$}

        \fig{0.5}{ex_fosvl1_12}{12}{Optical waveguide}

        \fig{0.5}{ex_fosvl1_13}{13}{optical glass fiber}

        \fig{0.5}{ex_fosvl1_14}{14}{Spherical lenses: combination of two seherical interfaces(air-glss \& glass-air) separated by a distance $\Delta$}

        \fig{0.5}{ex_fosvl1_15}{15}{}
        Conventions:
        \begin{itemize}
            \item $R>0(<0)$ for convex(concave) surface,
            \item $P_1(P_2)$ is measured in a leftward(rightward-) pointing coordinate system,
            \item $\theta$ is positive(negative) if the ray propagates into the direction of the positive(negative) y-axis
        \end{itemize}

        For thin lenses(i.e. $\Delta$ is negligible$\leftrightarrow$ y-coord. unchanged after lenses) and paraxial rays, we find
        $$
        \theta_2 = \theta_1 - \frac{y}{f},
        $$
        where the focal length is given by
        \begin{align}
            \frac{1}{f} = \left(n - 1\right)\left(\frac{1}{R_1} - \frac{1}{R_2}\right) \labtag{1.6}
        \end{align}
        which is lensmaker's equation

        \fig{0.5}{ex_fosvl1_16}{16}{}
        We have:
        \begin{itemize}
            \item lens equation
           \begin{align}
                \frac{1}{z_1} + \frac{1}{z_2} = \frac{1}{f}\labtag{1.7}
            \end{align}
            \item magnification
            \begin{align}
                y_2 = - \frac{z_2}{z_1}y_1 \labtag{1.8}
            \end{align}
        \end{itemize}

        Remarks:
        \begin{itemize}
            \item magnification$\left(-\frac{z_2}{z_1}\right)$ negative $\leftrightarrow$ image is flipped.
            \item Eq.\ref{tag: 1.6}-\ref{tag: 1.8} only hold for paraxial rays. Non-paraxial rays are subject to so-called spherical aberrations.
        \end{itemize}

    \subsection{Matrix optics}
        Matrix optics allows one to trace the propagation of paraxial rays through an optical system.

        {\bf 2-dim method $\Rightarrow$ can be applied to planes systems or to ??? rays in system with cylindrical? symmetry?}










\lecture{2}{20.11.12}{Prof. Arno Rauschenbeutel}{Guangyu He}{Fundamentals of Optical Sciences:experimental part}{WS 2020/21}
        Described ray by$(y,\theta)$
        \fig{0.5}{ex_fosvl2_1}{1}{}

        In paraxial approximation, we have $\sin\theta \simeq \theta$ \\
        thus, $(y_2,\theta_2)$ depends linearly on$(y_2,\theta_2)$, hence
        \begin{align}
            \left(\begin{matrix}
                y_2 \\
                \theta_2
            \end{matrix}\right) =  \left( \begin{matrix}
                A & B \\
                C & D
            \end{matrix}\right)\left(\begin{matrix}
                y_1 \\
                y_2 
            \end{matrix}\right)\labtag{1.9}
        \end{align} 
        where $u=\left(\begin{matrix}
            A & B \\
            C & D
        \end{matrix}\right)$ is ray transfer matrix.

        \underline{Matrices for simple optical components}
        \fig{0.5}{ex_fosvl2_2}{2}{Free-spcae propagation: We have $y_2 = y_1 + d\theta_1$ and $\theta_2 =\theta_1$. Thus, $u = \left(\begin{matrix}
            1 & d \\
            0 & 1
        \end{matrix}\right)$}

        \fig{0.5}{ex_fosvl2_3}{3}{Refraction at a plane interface: We have $y_1 = y_2$ and $n_1\theta_1 < n_2\theta_2$ (Snell's law in paraxial approximation). Thus, $ M = \left(\begin{matrix}
            1 & 0 \\
            0 & \frac{n_1}{n_2}
        \end{matrix}\right)$}

        \fig{0.5}{ex_fosvl2_4}{4}{Refraction at a spherical interface: Using the relation from the excercise sheet for $\theta_1 \& \theta_2$ and $y_1 = y_2$, it follows that $ M = \left(\begin{matrix}
            1 & 0 \\
            -\frac{n_1 - n_2}{n_2 R} & \frac{n_1}{n_2}
        \end{matrix}\right)$ where $R>0(<0)$ for convex(concave) lens}

        \fig{0.5}{ex_fosvl2_5}{5}{Propagation through a thin lens: Using the last result in conjunction with the lensmaker's equation\ref{tag: 1.6}, we have $\theta_2 = \theta_1 - y/R $ and $y_2 = y_1$. Thus, $ M = \left(\begin{matrix}
            1 & 0 \\
            -\frac{1}{f} & 1
        \end{matrix}\right)$ where $f>0(<0)$ for convex(concave) lens}

        \fig{0.5}{ex_fosvl2_6}{6}{Reflection at a plane mirror: \underline{Convention:} The z-axis always counts positive in the direction of propagation of the ray. Thus, $M = \left(\begin{matrix}
            1 & 0 \\
            0 & 1
        \end{matrix}\right)$}

        Reflection at a spherical mirror:\\
        Using the same convention and eq.\ref{tag: 1.4}, we find $ M = \left(\begin{matrix}
            1 & 0 \\
            2/R & 1
        \end{matrix}\right)$ which is analogous to a thin lens

        \underline{Matices for comlrned? optical components}\\
        Sequence of N optical components with matrices $M_1,M_2,\dots,M_N$ is equivalent to a single optical system with ray traces for matrix
        \begin{align}
            M = M_N\dots M_2 M_1 \labtag{1.10}
        \end{align}
        Note that the first matrix appears on the right!

        \fig{0.5}{ex_fosvl2_7}{7}{Example: Air gap plus thin lens}
        \begin{align}
            M = \left(\begin{matrix}
                1 & 0 \\
                -\frac{1}{f} & 1
            \end{matrix}\right) \left(\begin{matrix}
                1 & d \\
                0 & 1
            \end{matrix}\right) = \left(\begin{matrix}
                1 & d\\
                -\frac{1}{f} & 1 - \frac{d}{f}
            \end{matrix}\right) \nonumber
        \end{align}

        \underline{Remark:} As a formalism, matrix optics is way more important and powerful that what is apparent at this stage(labs?).

\section{Wave Optics}
    Light propagates in the form of a wave. In vacuum, the propagation velocity is
    \begin{align}
        c_0 = 3\times 10^8 m/s = 30 cm/ns = 0.3 mm/ps = 0.3 nm/fs \nonumber
    \end{align}

    Optical wavelengths comprise these ranges
    \fig{0.5}{ex_fosvl2_8}{8}{}
    N.B.:Ray optics is a limiting case of wave optics for infinitely small wavelengths.

    \subsection{The postulates of wave optics}
        \underline{Wave equation}

        Speed of light in vacuum: $c_0$

        Speed of light in a homogeneous transparent medium with index of refraction $n$:
        \begin{align}
            c = \frac{c_0}{n} \labtag{2.1}
        \end{align}

        mathematical description:\\
        wave function $u(\vecr,t)$(real)

        $u(\vecr,t)$ has to fulfill the wave equation
        \begin{align}
            \nabla^2 u - \frac{1}{u^2}\partialtt u = 0 \labtag{2.2}
        \end{align}

        N.B.: the wave equation is linear in $u \rightarrow$ superposition principle:\\
        If $u_1(\vecr,t)$ and $u_2(\vecr,t)$ represent valid optical waves their sum $u(\vecr,t) = u_1(\vecr,t) + u_2(\vecr,t)$ is also a valid optical wave.

        \underline{Intensity and power of optical waves}\\
        Intensity is optical power per unit area:
        \begin{align}
            I(\vecr,t) = 2 \langle u^2(\vecr,t) \rangle \labtag{2.3}
        \end{align}

        where $\langle \dots \rangle$ devotes averaging over a time much larger than the devotion of the optical period(e.g. $2\times 10^{-15}s$ for $\lambda = 600$nm)

        \underline{Remark:} Factor 2 in eq.\ref{tag: 2.3} is (still) arbitray because we (still) have not assigned a physical meaning to $u(\vecr,t)$.

        Let A be a surface that is oriented perpendicularly to the propagation direction of the light$\rightarrow$ optical pases impinging on A:
        \begin{align}
            P(t) = \int_A I(\vecr.t)dA \labtag{2.4}
        \end{align}  

    \subsection{homogeneous waves}

        \begin{align}
            u(\vecr,t) = a(\vecr)\cos(2\pi \nu t + \varphi(\vecr)) \labtag{2.5}
        \end{align}

        where $a(\vecr)$ is amplitude, $\varphi(\vecr)$ is phase, $\nu$ is frequency(Hz), \\
        $2\pi \nu =: \omega$ is angular frequency(rad$\cdot s^{-1}$)

        \underline{Complex notation}\\
        Very useful: complex representation of $u(\vecr,t)$ from eq.\ref{tag: 2.5}
        \begin{align}
            u(\vecr,t) = a(\vecr)e^{i\varphi(\vecr)}e^{i\omega t} \labtag{2.6}\nn
            \rightarrow u(\vecr,t) = \mathbf{R}\left\{ u(\vecr,t) \right\} = \frac{1}{2}(u + u^*) \labtag{2.7}
        \end{align}

        $u(\vecr,t)$ is complex wave function

        Wave equation:
        \begin{align}
            \nabla^2 u(\vecr,t) - \frac{1}{c}\partialtt u(\vecr,t) = 0 \labtag{2.8}
        \end{align}

        Note: $u(\vecr,t)$ have to fulfill the same boundary conditions.

        \underline{Complex amplitude}\\
        Rewrite
        \begin{align}
            u(\vecr,t) = u(\vecr) e^{i\omega t} \labtag{2.9}
        \end{align}

        where $u(\vecr) = a(\vecr) e^{i\varphi(\vecr)}$ is complex amplitude.

        N.B.: At a given position $\vecr$, $\abs{u(\vecr)} = a(\vecr)$ is the amplitude and $\arg(u(\vecr)) = \varphi(\vecr)$ the phase of the wave.

        \fig{0.5}{ex_fosvl2_9}{9}{}

        \underline{The Helmholtz equatuion}\\
        Insert $u(\vecr,t)$ from eq.\ref{tag: 2.3} into wave eq.\ref{tag: 2.8}, one gets Helmholtz equation
        \begin{align}
            \rightarrow (\nabla^2 + k^2)u(\vecr,t) = 0 \labtag{2.10}
        \end{align}
        where $k = \frac{2\pi \nu}{c} = \frac{\omega}{c}$ is wave number.

        \underline{Optical intensity}\\
        \begin{align}
            2 u^(\vecr,t) = 2a^2(\vecr) \cos^2(\omega t + \varphi(\vecr)) \nn
            = \abs{u(\vecr)}^2 \left[ 1 + \cos\left( 2(\omega t + \varphi(\vecr))\right)\right] \nonumber
        \end{align}

        where $\cos\left( 2(\omega t + \varphi(\vecr))\right) = 0 $when averaging
        \begin{align}
            \rightarrow I(\vecr) = \abs{u(\vecr)}^2 \labtag{2.11}
        \end{align}

        Optical intensity of a monochromatic wave is modules squared of the complex amplitude
        
        N.B.: The intensity of a monochromatic wave is constant in time.

        \underline{Wave fronts}\\
        Wave fronts is surface of equal phase
        \begin{align}
            \varphi(\vecr) = \text{const.} \labtag{2.12}
        \end{align}

        \fig{0.5}{ex_fosvl2_10}{10}{        e.g. $\varphi(\vecr) = 2\pi q $ with $q\in\mathbb{Z}$        }












\lecture{3}{20.11.13}{Prof. Arno Rauschenbeutel}{Guangyu He}{Fundamentals of Optical Sciences:experimental part}{WS 2020/21}

        \underline{Simple waves}\\
        \underline{Plane waves}\\
        Complex amplitude:
        \begin{align}
            u(\vecr) = Ae^{-i\veck\cdot\vecr} \nn
            = Ae^{-i(k_x x + k_y y + k_z z)} \labtag{2.13}
        \end{align}

        where $A = \text{const.} \in \mathbb{D}$ is complex envelope and $\veck = (k_x,k_y,k_z)$ is wave vector.

        Inserting eq.\ref{tag: 2.13} into Helmholtz eq\ref{tag: 2.10} $\rightarrow k_x^2 + k_y^2 + k_z^2 = k$ or
        \begin{align}
            \abs{\veck} = k = \frac{\omega}{c} \nonumber
        \end{align}

        Phase of the plane wave:
        \begin{align}
            \varphi(\vecr) = \arg(u(\vecr)) = \arg(A) - \veck\cdot \vecr \nonumber
        \end{align}
        $\rightarrow$ wave fronts: $\veck\cdot \vecr = 2\pi q + \arg(A)$\\
        this describes parallel phases $\bot$ to $\veck$

        ??? of two phases for $q \& q+1$:
        \begin{align}
            \lambda = \frac{2\pi}{k}  = \frac{c}{\nu} \labtag{2.15}
        \end{align}  

        where $\lambda$ is wavelength

        Example: Propagation along z-direction $\rightarrow u(\vecr) = Ae^{- i k z}$ and wave function
        \begin{align}
            u(\vecr,t) = \abs{A} \cos(\omega t - k z + \arg(A)) \nn
            = \abs{A} \cos(\omega(t - z/c) + \arg(A)) \nonumber
        \end{align}

        where $z/c$ is phase velocity.

        Un a medium with refractive index $n$, we have
        \begin{align}
            c = c_0 / n, \lambda = \lambda_0 / n, k = nk_0 \labtag{2.16}
        \end{align}

        N.B.: frequency remains unaltered
        \begin{align}
            \omega = ck=\frac{c_0}{n}\cdot nk_0 = c_0 k_0 = \omega_0 \nonumber
        \end{align}

        \underline{Spherical waves}\\
        complex amplitude in spherical coordinates:
        \begin{align}
            u(\vecr) = \frac{A_0}{r}e^{-ikr} \labtag{2.17}
        \end{align}
        where $r$ is distance from origin,$A_0 = \text{const.} \in \mathbb{D}$
        \begin{align}
            \rightarrow I(\vecr) = \frac{\abs{A_0}^2}{r}\labtag{2.18}
        \end{align}

        For $\arg(A_0) = 0$, the wave fronts are concentric spherical shells with radius $r = 2\pi q/k = q\lambda$, which are thus separated by $\lambda$.

        N.B.:
        \begin{itemize}
            \item Spherical wave issuing from $\vecr_0$:
            \begin{align}
                u(\vecr) = \left(\frac{A_0}{\abs{\vecr - \vecr_0}}\right)e^{-ik\abs{\vecr-\vecr_0}} \nonumber
            \end{align}
            \item For eq.\ref{tag: 2.17}, the spherical wave propagates towards the origin rather than away from it.
        \end{itemize}

        \underline{Fresnela approximation for spherical waves(paralotic? waves)}\\
        Consider spherical wave for away from origin but near the optical axis
        \begin{align}
            \rightarrow \sqrt{x^2 + y^2} \ll z \text{   or  } \theta^2 = \frac{x^2 + y^2}{z^2} \ll 1 \nonumber
        \end{align}

        Using the Taylor expansion $\sqrt{1 + \varepsilon} \simeq 1 + \frac{\varepsilon}{2}$ for $\varepsilon \ll 1$, we can write
        \begin{align}
            r = \sqrt{x^2 + y^2+ z^2} = z\sqrt{ 1 + \theta^2} \simeq z\left(1 + \frac{\theta^2}{2}\right) = z + \frac{x^2 + y^2}{2z} \labtag{2.19}
        \end{align}

        and thus,
        \begin{align}
            u(\vecr) = \frac{A_0}{z}e^{-ikz}e^{-ik\frac{x^2 + y^2}{2z}} \labtag{2.20}
        \end{align}

        N.B.:
        \begin{itemize}
            \item Magnitude is less susceptible to errors that the phase, where awe require that the error in $kr$ is much smaller than $2\pi$, or that the error in $r$ is $\ll 2\pi/k = \lambda$. There, we used $r = z\sqrt{1 + \theta^2} \simeq z$ in the denominator.
            \item eq.\ref{tag: 2.20} is an important result for describing diffraction(???).
        \end{itemize}

        \underline{Validity of Fresnel Approximation}\\
        Next higher order ??? in Taylor expansion of $\sqrt{1 + \theta^2} $ reads $ - \frac{\theta^4}{8}$. In order to neglect it, we must have
        \begin{align}
            kz\theta^4/ 8 \ll \pi \text{ or } (x^2 + y^2)^2 \ll 4z^3\lambda \text{ or } a^4 \ll 4z^3\lambda \nonumber
        \end{align}

        where $a$ is the distance of the point $(x,y)$ from the z-axis(optical axis).

        Thus, for all points$(x,y)$, we must have
        \begin{align}
            \frac{\text{N}_\text{F} \theta^2_\text{max}}{4} \ll 1 \labtag{2.21}
        \end{align}

        where $\theta_\text{max} = \frac{a_\text{max}}{z} $ is the maximum angle and Fresnel number:
        \begin{align}
            \text{N}_\text{F} = \frac{a^2_\text{max}}{\lambda z} \labtag{2.22}
        \end{align}

        \underline{Paraxial waves} \\
        Wave paraxial $\leftrightarrow$ normals to wave fronts are paraxial rays $\leftrightarrow$ complex amplitudes is a modulated plane wave
        \begin{align}
            u(\vecr) = A(\vecr)e^{-ikz} \labtag{2.23}
        \end{align}

        where the variation of $A(\vecr)$ and of $\frac{\partial A(\vecr)}{\partial z }$ has to ve sufficiently small on the scale of the wavelength $\lambda = \frac{2\pi}{k}$

        \underline{Paraxial Helmoholtz equation}\\
        Quantitative formulation of the condition for paraxial waves:
        \begin{itemize}
            \item Change of $A(\vecr)$ within interval $\Delta z = \lambda $ has to be much smaller than $A(\vecr)$ itself, i.e.,
            \begin{align}
                \frac{\partial A}{\partial z} \cdot \lambda \ll A \text{ or } \frac{\partial A}{\partial z} \ll kA \labtag{2.24}
            \end{align}
            \item Change of $\partial A / \partial z$ within interval $\Delta z = \lambda$ has to be much smaller than $\partial A / \partial z$ itself, i.e.,
            \begin{align}
                \frac{\partial^2 A}{\partial z^2} \ll k \frac{\partial A}{\partial z} \labtag{2.25}
            \end{align}
            Thus, in particular
            \begin{align}
                \frac{\partial^2 A}{\partial z^2} \ll k^2 A \labtag{2.26}
            \end{align}
            Inserting eq\ref{tag: 2.23} into Helmholtz eq\ref{tag: 2.10} and neglecting $\partial^2 A / \partial z^2$ ??? $k\cdot \partial A / \partial z$ and $k^2A$, We obtain(see exercises) paraxial Helmholtz equation:
            \begin{align}
                \Delta_t^2 A - 2 ik\frac{\partial A}{\partial z} = 0 \labtag{2.27}
            \end{align}
            where $\Delta_t^2 = \frac{\partial^2}{\partial x^2} + \frac{\partial^2}{\partial y^2}$: Transverse Laplace operator. 
        \end{itemize}

        \underline{Remarks:}
        \begin{itemize}
            \item eq.\ref{tag: 2.25} and \ref{tag: 2.26}: slowly varying envelope approximation(SVEA),
            \item eq.\ref{tag: 2.27} is formally equivalent to Schrödinger equation in quantum mechanics,
            \item Amongst others, eq.\ref{tag: 2.27} is solved by 
            \begin{itemize}
                \item parabolic waves
                \item gaussian beams(important/useful for optical Resonators, Lasers, etc.)
            \end{itemize}
        \end{itemize}
\end{document}