%last updated 01.11.19 by GuangyuHe

\documentclass[twoside]{article}
%\documentclass[UTF8]{ctexart}
\setlength{\oddsidemargin}{0.25 in}
\setlength{\evensidemargin}{-0.25 in}
\setlength{\topmargin}{-0.6 in}
\setlength{\textwidth}{6.5 in}
\setlength{\textheight}{8.5 in}
\setlength{\headsep}{0.75 in}
\setlength{\parindent}{0 in}
\setlength{\parskip}{0.1 in}

%
% ADD PACKAGES here:
%

\usepackage[UTF8]{ctex}

\usepackage{amsmath,amsfonts,graphicx}
\usepackage{esint}
\usepackage{multicol}
%\usepackage{MnSymbol}
\usepackage{mathtools}
\usepackage[bottom]{footmisc}
\usepackage[pdfpagelabels,hyperindex,hyperfigures,breaklinks,colorlinks,linkcolor=black,citecolor=black,urlcolor=black]{hyperref}
\usepackage{geometry}
\usepackage{marginnote}
\usepackage{color}
%
% The following commands set up the lecnum (lecture number)
% counter and make various numbering schemes work relative
% to the lecture number.
%

\newcounter{lecnum}
\renewcommand{\thepage}{\thelecnum-\arabic{page}}
%\renewcommand{\thesection}{\thelecnum.\arabic{section}}
%\renewcommand{\theequation}{\thelecnum.\arabic{equation}}
%\renewcommand{\thefigure}{\thelecnum.\arabic{figure}}
%\renewcommand{\thetable}{\thelecnum.\arabic{table}}

%
% The following macro is used to generate the header.
%
\newcommand{\lecture}[6]{
   \pagestyle{myheadings}
   \thispagestyle{plain}
   \newpage
   \setcounter{lecnum}{#1}
   \setcounter{page}{1}
   \noindent
   \begin{center}
   \framebox{
      \vbox{\vspace{2mm}
    \hbox to 6.28in { {\bf #5
	\hfill #6} }
       \vspace{4mm}
       \hbox to 6.28in { {\Large \hfill Lecture #1: #2  \hfill} }
       \vspace{2mm}
       \hbox to 6.28in { {\it Lecturer: #3 \hfill Scribes: #4} }
      \vspace{2mm}}
   }
   \end{center}
   \markboth{Lecture #1: #2}{Lecture #1: #2}

   %{\bf Note}: {\it LaTeX template courtesy of UC Berkeley EECS dept.}

   %{\bf Disclaimer}: {\it These notes have not been subjected to the
   %usual scrutiny reserved for formal publications.  They may be distributed
  % outside this class only with the permission of the Instructor.}
   \vspace*{4mm}
}

\newcommand{\exercise}[6]{
   \pagestyle{myheadings}
   \thispagestyle{plain}
   \newpage
   \setcounter{lecnum}{#1}
   \setcounter{page}{1}
   \noindent
   \begin{center}
   \framebox{
      \vbox{\vspace{2mm}
    \hbox to 6.28in { {\bf #5
	\hfill #6} }
       \vspace{4mm}
       \hbox to 6.28in { {\Large \hfill Exercise #1: #2  \hfill} }
       \vspace{2mm}
       \hbox to 6.28in { {\it Lecturer: #3 \hfill Scribes: #4} }
      \vspace{2mm}}
   }
   \end{center}
   \markboth{Exercise #1: #2}{Exercise #1: #2}
   
   %{\bf Note}: {\it LaTeX template courtesy of UC Berkeley EECS dept.}

   %{\bf Disclaimer}: {\it These notes have not been subjected to the
   %usual scrutiny reserved for formal publications.  They may be distributed
  % outside this class only with the permission of the Instructor.}
  \vspace*{4mm}
}

\newcommand{\review}[4]{
   \pagestyle{myheadings}
   \thispagestyle{plain}
   \newpage
   %\setcounter{lecnum}{#1}
   \setcounter{page}{1}
   \noindent
   \begin{center}
   \framebox{
      \vbox{\vspace{2mm}
    \hbox to 6.28in { {\bf #3
	\hfill #4} }
       \vspace{4mm}
       \hbox to 6.28in { {\Large \hfill Review \hfill} }
       \vspace{2mm}
       \hbox to 6.28in { {\it Lecturer: #1 \hfill Scribes: #2} }
      \vspace{2mm}}
   }
   \end{center}
   \markboth{Review}{Review}
   
   %{\bf Note}: {\it LaTeX template courtesy of UC Berkeley EECS dept.}

   %{\bf Disclaimer}: {\it These notes have not been subjected to the
   %usual scrutiny reserved for formal publications.  They may be distributed
  % outside this class only with the permission of the Instructor.}
  \vspace*{4mm}
}

\newcommand{\answer}[6]{
   \pagestyle{myheadings}
   \thispagestyle{plain}
   \newpage
   \setcounter{lecnum}{#1}
   \setcounter{page}{1}
   \noindent
   \begin{center}
   \framebox{
      \vbox{\vspace{2mm}
    \hbox to 6.28in { {\bf #5
	\hfill #6} }
       \vspace{4mm}
       \hbox to 6.28in { {\Large \hfill Answer Sheet #1: #2  \hfill} }
       \vspace{2mm}
       \hbox to 6.28in { {\hfill #3 , #4} }
      \vspace{2mm}}
   }
   \end{center}
   \markboth{Answer Sheet #1: #2}{Answer Sheet #1: #2}
   
   %{\bf Note}: {\it LaTeX template courtesy of UC Berkeley EECS dept.}

   %{\bf Disclaimer}: {\it These notes have not been subjected to the
   %usual scrutiny reserved for formal publications.  They may be distributed
  % outside this class only with the permission of the Instructor.}
  \vspace*{4mm}
}


%
% Convention for citations is authors' initials followed by the year.
% For example, to cite a paper by Leighton and Maggs you would type
% \cite{LM89}, and to cite a paper by Strassen you would type \cite{S69}.
% (To avoid bibliography problems, for now we redefine the \cite command.)
% Also commands that create a suitable format for the reference list.
\renewcommand{\cite}[1]{[#1]}
\def\beginrefs{\begin{list}%
        {[\arabic{equation}]}{\usecounter{equation}
         \setlength{\leftmargin}{2.0truecm}\setlength{\labelsep}{0.4truecm}%
         \setlength{\labelwidth}{1.6truecm}}}
\def\endrefs{\end{list}}
\def\bibentry#1{\item[\hbox{[#1]}]}

%Use this command for a figure; it puts a figure in wherever you want it.
%usage: \fig{NUMBER}{CAPTION}{picture scale}{picture name}
\newcommand{\fig}[4]{
    \begin{center}
        \includegraphics[scale=#1]{fig//#2}
    \end{center}
    \begin{center}
        Figure #3 #4
    \end{center}


    %\begin{center}
    %    \includegraphics[scale=#3]{#4}
    %\end{center}
    %\vspace{2mm}
    %\begin{center}            
    %    Figure \thelecnum.#1:~#2
    %\end{center}
}
% Use these for theorems, lemmas, proofs, etc.
%\newtheorem{theorem}{Theorem}[lecnum]
%\newtheorem{lemma}[theorem]{Lemma}
%\newtheorem{proposition}[theorem]{Proposition}
%\newtheorem{claim}[theorem]{Claim}
%\newtheorem{corollary}[theorem]{Corollary}
%\newtheorem{definition}[theorem]{Definition}
%\newenvironment{proof}{{\bf Proof:}}{\hfill\rule{2mm}{2mm}}

% **** IF YOU WANT TO DEFINE ADDITIONAL MACROS FOR YOURSELF, PUT THEM HERE:

\newcommand\E{\mathbb{E}}

\newcommand{\nn}{
    \nonumber \\
}%不带公式编号换行

\newcommand{\ketn}[1]{
    | #1 \rangle
}%右矢

\newcommand{\bran}[1]{
    \langle #1 |
}%左矢

\newcommand{\braoket}[1]{
    \langle #1 | #1 \rangle
}

\newcommand{\bratket}[2]{
    \langle #1 | #2 \rangle
}

\newcommand{\inpro}[2]{
    \langle #1 , #2 \rangle
}%内积

\newcommand{\rightarrowtxt}[1]{
    \stackrel{\text{#1}}{\longrightarrow}
}%箭头上带文字

\newcommand{\ahat}{
    \hat{a}
}%下降算符

\newcommand{\adag}{
    \hat{a}^\dagger
}%上升算符

\newcommand{\osrt}{
    \frac{1}{\sqrt{2}}
}%根二分之一

\newcommand{\abs}[1]{
    \left| #1 \right|
}%绝对值

\newcommand{\sumint}{
    \mathclap{\displaystyle\int}\mathclap{\textstyle\sum}
}%叠加积分

\newcommand{\longline}{
    \rule{\textwidth}{0.5mm}
} %页面分割线

\newcommand{\explain}[1]{
    \footnote{{\it #1}}
}%脚注+斜体说明

\newcommand{\sidemark}[1]{
    \reversemarginpar
    \marginnote{\textsl{side remark:} \\ #1 }[3cm]
}%侧边注

\newcommand{\highlightquestion}[1]{
    {\color{red} #1 }
}%红色高亮问题

\newcommand{\highlight}[1]{
    {\color{blue} #1 }
}%蓝色高亮重点

\newcommand{\sbeq}{
    \overset{!}{=}
}%should be equal

\newcommand{\bigkuohao}[2]{
    \left\{
    \begin{aligned}
   #1 \\
    #2
    \end{aligned}
    \right.
}

\newcommand{\txtoneq}[1]{
    \stackrel{\text{#1}}{=}
}%等号上带文字

\newcommand{\hate}{
    \hat{e}
}

\newcommand{\veca}{
    \vec{A}
}

\newcommand{\vecb}{
    \vec{B}
}

\newcommand{\vecd}{
    \vec{D}
}

\newcommand{\vece}{
    \vec{E}
}

\newcommand{\vech}{
    \vec{H}
}

\newcommand{\vecj}{
    \vec{J}
}

\newcommand{\vecjj}{
    \vec{j}
}

\newcommand{\veck}{
    \vec{k}
}

\newcommand{\vecp}{
    \vec{P}
}

\newcommand{\vecpp}{
    \vec{p}
}

\newcommand{\vecs}{
    \vec{S}
}

\newcommand{\vecr}{
    \vec{r}
}

\newcommand{\vecnabla}{
    \vec{\nabla}
}

\newcommand{\partialt}{
    \frac{\partial}{\partial t}
}

\newcommand{\partialtt}{
    \frac{\partial^2}{\partial t^2}
}

\newcommand{\labtag}[1]{
    \label{tag: #1}\tag{#1}
}







%command for physics uses


%\begin{document}
    % Some general latex examples and examples making use of the
% macros follow.  
%**** IN GENERAL, BE BRIEF. LONG SCRIBE NOTES, NO MATTER HOW WELL WRITTEN,
%**** ARE NEVER READ BY ANYBODY.

%\section{Some theorems and stuff} % Don't be this informal in your notes!

%\begin{lemma} %引理
%This is the first lemma of the lecture.
%\end{lemma}

%\begin{proof} %证明
%The proof is by induction on $\ldots$.
%For fun, we throw in a figure.
%%%NOTE USAGE !
%\fig{1}{1in}{A Fun Figure} %插入图片说明

%This is the end of the proof, which is marked with a little box.
%\end{proof}

%\subsection{A few items of note}

%Here is an itemized list:
%\begin{itemize}
%\item this is the first item;
%\item this is the second item.
%\end{itemize}

%Here is an enumerated list:
%\begin{enumerate}
%\item this is the first item;
%\item this is the second item.
%\end{enumerate}

%Here is an exercise:

%{\bf Exercise:}  Show that ${\rm P}\ne{\rm NP}$.

%Here is how to define things in the proper mathematical style.
%Let $f_k$ be the $AND-OR$ function, defined by

%\[ f_k(x_1, x_2, \ldots, x_{2^k}) = \left\{ \begin{array}{ll}

%	x_1 & \mbox{if $k = 0$;} \\

%	AND(f_{k-1}(x_1, \ldots, x_{2^{k-1}}),
%	   f_{k-1}(x_{2^{k-1} + 1}, \ldots, x_{2^k}))
%	 & \mbox{if $k$ is even;} \\

%	OR(f_{k-1}(x_1, \ldots, x_{2^{k-1}}),
%	   f_{k-1}(x_{2^{k-1} + 1}, \ldots, x_{2^k}))	
%	& \mbox{otherwise.} 
%	\end{array}
%	\right. \]

%\begin{theorem}
%This is the first theorem.
%\end{theorem}

%\begin{proof}
%This is the proof of the first theorem. We show how to write pseudo-code now.
%*** USE PSEUDO-CODE ONLY IF IT IS CLEARER THAN AN ENGLISH DESCRIPTION

%Consider a comparison between $x$ and~$y$:
%\begin{tabbing}
%\hspace*{.25in} \= \hspace*{.25in} \= \hspace*{.25in} \= \hspace*{.25in} \= \hspace*{.25in} \=\kill
%\>{\bf if} $x$ or $y$ or both are in $S$ {\bf then } \\
%\>\> answer accordingly \\
%\>{\bf else} \\
%\>\>    Make the element with the larger score (say $x$) win the comparison \\
%\>\> {\bf if} $F(x) + F(y) < \frac{n}{t-1}$ {\bf then} \\%
%\>\>\> $F(x) \leftarrow F(x) + F(y)$ \\
%\>\>\> $F(y) \leftarrow 0$ \\
%\>\> {\bf else}  \\
%\>\>\> $S \leftarrow S \cup \{ x \} $ \\
%\>\>\> $r \leftarrow r+1$ \\
%\>\> {\bf endif} \\
%\>{\bf endif} 
%\end{tabbing}

%This concludes the proof.
%\end{proof}


%\section{Next topic}

%Here is a citation, just for fun~\cite{CW87}.

%\section*{References}
%\beginrefs
%\bibentry{CW87}{\sc D.~Coppersmith} and {\sc S.~Winograd}, 
%``Matrix multiplication via arithmetic progressions,''
%{\it Proceedings of the 19th ACM Symposium on Theory of Computing},
%1987, pp.~1--6.
%\endrefs

% **** THIS ENDS THE EXAMPLES. DON'T DELETE THE FOLLOWING LINE:
%\end{document}
\begin{document}
    \lecture{1}{20.11.04}{Prof. Alejandro Saenz}{Guangyu He}{Fundamentals of Optical Sciences:theory part}{WS 2020/21}

    \setcounter{section}{-1}
    \section{General remarks}
        Exercises: 50\% handed via Moodle in hand-written PDF

        (most-likely) Oral(digital) exam

        Communication via Moodle make sure you check your e-mail account to which Moodle sends e-mails!

        \underline{Content(theory part):}

        I. Basic Electrodynamics,

        II. Light-atom interaction in the semi-classical picture(description),

        III. Quantization of electromagnetic fields,
        \# 09:53-09:57






    \section{Basic Electrodynamics}
    \setcounter{subsection}{-1}
        \subsection{Maxwell equations}
            $$
            \begin{aligned}
                \vecnabla \cdot \vec{D} &= \rho_f \nn
                \vecnabla \cdot \vec{B} &= 0 \nn
                \vecnabla \times \vec{E} &= - \partialt \vec{B} \nn
                \vecnabla \times \vec{H} &= \partialt \vec{D} + \vec{j}_f \nn
            \end{aligned}
            $$

            and:
            $$
            \begin{aligned}
                \vec{D} = \varepsilon \vec{E} + \vec{P} \nn
                \vec{H} = \frac{1}{\mu} \vec{B} + \vec{M} \nn
            \end{aligned}
            $$

            for $\vec{P} = 0$:
            $$
            \vec{D} = \varepsilon \vec{E}
            $$

            where $\varepsilon = \varepsilon_0\left(1+\chi_e\right)$

            alterative: $\vec{D} = \varepsilon_0 \cdot \varepsilon_p \cdot \vec{E} $

            {\sl Don't mess up source!}

            \longline

            Lorentz force law: \# 10:15-10:20
            $$
            \vec{F} = q \left(\vec{E} + \vec{v} \times \vec{B}\right)
            $$
            {\sl Setting charge particals and back field}






        \subsection{Potential picture and gauge transformations}
            \# 10:21

            While due to Faraday's law the electric field can not be written as the gradient of a (scalar) potential as in
            electrostatics, the magnetic field can still be written as the curl of a vector potential:

            \begin{align}
                \vec{B} = \vec{\nabla} \times \vec{A}  \labtag{I.1} \nonumber
            \end{align}

            Insertion of \ref{tag: I.1} into Faraday's law yields
            \begin{align}
                \vecnabla \times \vec{E} = - \partialt{B} \txtoneq{(I.1)} - \partialt\left(\vec{\nabla}\times \vec{A}\right) \nonumber \\
                \rightarrow   \vecnabla \times \vec{E} + \vecnabla \times \left(\partialt \vec{A}\right) = 0 \nonumber \\
                \rightarrow \vecnabla \times \left(\vec{E} + \partialt \vec{A}\right) = 0 \labtag{I.2}
            \end{align}

            \# 10:30

            we thus obtain a new quatitiy with a vanishing curl which thus can be written as the 
            gradient of a scalar potential:
            \begin{align}
                \rightarrow \vec{E} + \partialt\vec{A} =: - \vec{\nabla}\phi \labtag{I.3} \nonumber
            \end{align}
            (note, the choice of the negative sign is only convention)

            This yields in turn for the electric fields:
            \begin{align}
                \vec{E} = - \left(\vecnabla \phi + \partialt \vec{A} \right) \labtag{I.4}
            \end{align}

            Note, when using the scalar potential$\phi$ and the vector potential $\vec{A}$, two of 
            the Maxwell equations are automatically fulfilled, as they were used in the deviations of 
            the potentials. $\left(\vec{\nabla}\cdot\vec{B} = 0, \vecnabla \times \vec{E} = -\partialt \vec{B}\right)$

            \longline

            Potential picture / description: using $\phi$ and $\vec{A}$

            Field picture / description: using $ \vec{E} $ and $\vec{B}$

            Maxwell equations in potential picture:
            \begin{align}
                \text{(i) Gauss:    } \vecnabla \cdot \vec{E} = \frac{1}{\varepsilon_0} \rho \nn
                \rightarrow - \vec{\nabla} \left(\vec{\nabla}\phi + \partialt \vec{A} \right) = \frac{1}{\varepsilon_0} \rho \nn
                \vecnabla^2\phi + \partialt\left(\vec{\nabla}\cdot\vec{A}\right) =  -  \frac{1}{\varepsilon_0} \rho  \labtag{I.5}
            \end{align}










            \lecture{2}{20.11.05}{Prof. Alejandro Saenz}{Guangyu He}{Fundamentals of Optical Sciences:theory part}{WS 2020/21}

            \begin{align}
                \text{(ii) Ampere: }  \vecnabla \times \vec{B} &= \mu_0 \vec{j} + \mu_0 \varepsilon_0 \partialt \vec{E} \nn
                \vec{\nabla} \times \left(\vec{\nabla} \times \vec{A}\right)  &= \mu_0 \vec{j} + \mu_0 \varepsilon_0 \left(\vec{\nabla} \phi + \partialt \vec{A}\right) \nn
                &=  \mu_0 \vec{j} + \mu_0 \varepsilon_0  \left(\vec{\nabla} \partialt \phi + \partialtt \vec{A}\right) \nonumber
            \end{align}
            
            Remind: $ \vecnabla \times \left(\vecnabla \times \vec{f}\right) = \vec{\nabla} \left(\vecnabla \cdot \vec{f}\right) - \nabla^2 \vec{f}$

            \begin{align}
                \rightarrow \vec{\nabla}\cdot \left(\vec{\nabla}\cdot \vec{A}\right) - \nabla^2 \vec{A} = \mu_0 \vec{j} - \mu_0\varepsilon_0\left(\vec{\nabla}\partialt\phi + \partialtt \vec{A}\right) \nn
                \left(\nabla^2 \vec{A} - \mu_0\varepsilon_0 \partialtt\vec{A}\right) - \vec{\nabla}\left(\vec{\nabla}\cdot \vec{A} + \mu_0\varepsilon_0 \partialt \phi\right) = -\mu_0 \vec{j} \labtag{I.6}
            \end{align}

            Besides the fact that the equations determining the potentials do not look very nice, the problem of finding 6 field components has reduced to 4 potential components.

            $E_x,E_y,E_z,B_x,B_y,B_z \longrightarrow A_x,A_y,A_z,\phi$

            {\sl field can be measured and unique, but potential has a kind of freedom.}

            Already from electro- and magnetostatics it is , however, known that the definition of the potentials for give E and B fields is not unique!

            Which freedom in choosing the potentials do we have in electrodynamics:

            \begin{align}
                \vec{A}^\prime = \vec{A} + \vec{a} \nn
                \phi^\prime = \phi + f \nonumber
            \end{align}    

            using \ref{tag: I.1}, i.e. $\vec{B} = \vec{\nabla}\times\vec{A}$,we found
            $$
            \vecnabla \times \vec{A}^\prime \txtoneq{!} \vec{\nabla}\times\vec{A} \Rightarrow \vec{\nabla}\times\vec{a} = 0 \Rightarrow \vec{a} = \vec{\nabla}\phi 
            $$

            using \ref{tag: I.4}, i.e. $\vec{E} = - \left(\vec{\nabla}\phi + \partialt \vec{A}\right)$
            \begin{align}
                \vec{\nabla}\phi^\prime + \partialt \vec{A}^\prime &\txtoneq{!} \vec{\nabla}\phi + \partialt\vec{A}\nn
                \vec{\nabla}\phi + \vec{\nabla}f + \partialt\vec{A} + \partialt\vec{a} &\txtoneq{!} \vec{\nabla} \partialt\vec{A}\nn
                \vecnabla f + \partialt\vec{a} = 0 \Rightarrow \vec{\nabla}f + \partialt\left(\vec{\nabla}\phi\right) &= 0\nn
                \Rightarrow \vec{\nabla} \left(f + \partialt \phi\right) &= 0\nn
                \Rightarrow f &= - \partialt \phi \labtag{I.7}
            \end{align}
            \# 09:43 constant zero

            in fact, the result remian unchanged, if we have $f^\prime = f + c(t)$
            \# 09:45

            and gauge freedom: \# 09:47
            \begin{align}
                \vec{A}^\prime = \vec{A} + \vec{\nabla}\phi \nn
                \phi^\prime = \phi - \partialt \phi \labtag{I.8}
            \end{align}
            
            In other words, the two vector potentials $\vec{A}$ and$\vec{A}^\prime$ yields the same E and B field, if the scalar potential $\phi$ is changed accordingly, and vice versa.

            Equation \ref{tag: I.8} defines thus so-called gauge transformations, the freedom to choose the gauge potential $\phi$ is known as gauge invariance(or gauge freedom).
            
            \# 09:52-09:57

            Where there is in principle an infinite number of possible gauge transformations, tweo special choices for the gauge potential have proven most popular.

            \underline{(I)Coulomb gauge}

            Choice of the potentials such that
            $$
            \vec{\nabla}\cdot \vec{A} \txtoneq{!} 0
            $$
            with this choice one obtains
            \begin{align}
                \vec{\nabla}^2 \phi + \partialt\left(\vecnabla \cdot \vec{A}\right) \txtoneq{\ref{tag: I.5}}  - \frac{\rho}{\varepsilon_0}  \Rightarrow \vec{\nabla}^2 \phi =  - \frac{\rho}{\varepsilon_0} \labtag{I.9} \nonumber
            \end{align}

            which is identical to the Poisson equation know from electrostatics.

            However, for the vector potential, only one term in Equation \ref{tag: I.6} vanishes:
            \begin{align}
                \left(\vec{\nabla}^2\vec{A} - \mu_0 \varepsilon_0 \partialtt \vec{A}\right) = -\mu_0 \vec{j} + \mu_0\varepsilon_0 \vec{\nabla}\left(\partialt \phi\right) \nonumber
            \end{align}

            introducing the d'Alembert operator \# 10:09
            \begin{align}
                \square := \vec{\nabla}^2 - \mu_0 \varepsilon_0 \partialtt = \Delta - \frac{1}{c^2}\partialtt \nonumber
            \end{align}

            we obtain
            \begin{align}
                \square \vec{A}=-\mu_0\vec{j} + \mu_0\varepsilon_0\vec{\nabla}\left(\partialt \phi\right) \labtag{I.10} \nonumber
            \end{align}
            
            \# 10:13
            {\sl using only in vacuum}

            Note: in fact the d'Alembert operator is, in practice not always uniquely defined the way given above, but in a linear, homogeneous
            medium one implies $\varepsilon\mu$ instead of $\varepsilon_0\mu_0$.

            (In fact it can be shown that the vector potential in Coulomb gauge depends solely on the transversal component of the current.)

            \underline{(II) Lorenz gauge}
            \begin{align}
                \vec{\nabla}\cdot \vec{A} \txtoneq{!} -\mu_0 \varepsilon_0 \partialt \phi \nonumber
            \end{align}
            From Eq.\ref{tag: I.5} one gets:
            \begin{align}
                \vec{\nabla}^2 \phi + \partialt\left(\vec{\nabla}\cdot \vec{A}\right) = -\frac{\rho}{\varepsilon_0} \nn
                \Rightarrow \vec{\nabla}^2 \phi - \mu_0\varepsilon_0 \partialtt\phi = -\frac{\rho}{\varepsilon_0} \nn
                \rightarrow \square\phi = -\frac{\rho}{\varepsilon_0} \labtag{I.11} \nonumber
            \end{align}

            and with Eq.\ref{tag: I.6}:
            \begin{align}
                \left(\vec{\nabla}^2\vec{A} -  \mu_0\varepsilon_0 \partialtt\vec{A}\right) - \vec{\nabla}\left(\vec{\nabla}\cdot \vec{A} +  \mu_0\varepsilon_0 \partialt\phi\right) = - \mu_0 \vec{j} \nn
                \rightarrow \square \vec{A} = -\mu_0 \vec{j} \labtag{I.12} \nonumber
            \end{align}
            \# 10:35-10:38 different between two gauge

            Note: in Lorenz gauge the potentials $\vec{A}$ and $\phi$ are completely decoupled and one depends only on the charge density,
            the other only on the current.\\
            Furthermore, note the symmetry of the determining equations for the scalar and the vector potentials.

            While for computations, either gauge may be more efficient, "easier", ..., the Lorenz gauge is typically more appropriate for a deeper understanding of electrodynamics and relativity.
            
            Important: in fact the Lorenz gauge condition does not define the scalar and vector potential uniquely! Thus the Lorenz gauge is actually a gauge class!










            \lecture{3}{20.11.06}{Prof. Alejandro Saenz}{Guangyu He}{Fundamentals of Optical Sciences:theory part}{WS 2020/21}

            The equation of motion of charged particles in electromagnetic fields is guided by the Lorentz force law that evidently also can be expressed in terms of the potentials(instead of fields):

            \begin{align}
                \vec{F} = \frac{d\vec{p}}{d\vec{t}} = q\cdot \left(\vec{E} + \vec{v}\times\vecb\right) \labtag{I.13} \nonumber
            \end{align}

            where $\vec{F}$: force, $\vec{p}$: momentum, $\vec{v}$: velocity, q: charge of the particles.

            \begin{align}
                \rightarrow \vec{F} = q\cdot \left[-\vecnabla \phi - \partialt \vec{A} + \vec{v} \times \left(\vecnabla \times \vec{A}\right)\right] \labtag{I.14} \nonumber
            \end{align}

            using the product rule: $ \vec{\nabla}\cdot\left(\vec{f}\cdot \vec{g}\right) = \vec{f}\times\left(\vecnabla\times\vec{g}\right) + \vec{g} \times \left(\vecnabla\times\vec{f}\right) + \left(\vec{f}\cdot \vec{\nabla}\right)\cdot \vec{g} + \left(\vec{g}\cdot \vecnabla\right)\cdot \vec{f} $

            (here$\vec{f}\doteq \vec{v}; \vec{g}\doteq \vec{A}$)

            \begin{align}
                \vec{v}\times\left(\vec{\nabla}\times\vec{A}\right) = \vecnabla\cdot\left(\vec{v}\cdot\vec{A}\right) - \vec{A}\times \left(\vecnabla\times\vec{v}\right)-\left(\vec{v}\cdot \vec{\nabla}\right)\cdot \vec{A} - \left(\vec{A}\cdot\vecnabla\right)\cdot \vec{v} \labtag{I.15}
            \end{align}

            where
            \begin{align}
                \vecnabla\times\vec{A}: \vec{v} = \frac{d_x}{dt}\hate_x + \frac{d_y}{dt}\hate_y + \frac{d_z}{dt}\hate_z \nn
                \rightarrow \vecnabla \times \veca \rightarrow \text{ term of the types} \nn
                \frac{d}{d_{x_i}} \cdot \frac{d_{x_j}}{dt} = \frac{d}{dt} \frac{d_{x_j}}{d_{x_i}} = 0 \text{ for }i\neq j \nn
                \rightarrow \vecnabla \times \veca = 0 \nonumber
            \end{align}

            and 
            \begin{align}
                \vec{A}\cdot \vec{\nabla} = A_x \cdot \frac{d}{d_x} + A_y \cdot \frac{d}{d_y} + A_z \cdot \frac{d}{d_z} \nn
                \left(\veca\cdot\vecnabla\right)\cdot\vec{v} = A_x \frac{d}{d_x}\frac{d_x}{dt} + \cdots \nn
                = \frac{d}{dt}\frac{d_x}{d_x} = 0 \nonumber
            \end{align}

            so with \ref{tag: I.15}
            \begin{align}
                \vec{v}\times\left(\vecnabla\times\veca\right) = \vecnabla \cdot \left(\vec{v}\cdot \veca\right) - \left(\vec{v}\cdot \vecnabla\right)\cdot \vec{A} \labtag{I.16}\nonumber
            \end{align}

            and with \ref{tag: I.14}
            \begin{align}
                \vec{F} = q \cdot \left[-\vecnabla \phi - \partialt \veca + \vec{\nabla}\cdot \left(\vec{v}\cdot \veca\right) - \left(\vec{v}\cdot\vecnabla\right)\cdot\veca    \right] \nn
                = -q \left[\partialt \veca +\left(\vec{v} \cdot \vecnabla\right)\cdot \vec{A} + \vecnabla \left(\phi - \vec{v}\cdot\vec{A}\right)\right] \labtag{I.17} \nonumber
            \end{align}

            \# 11:44
            where $\partialt \veca +\left(\vec{v} \cdot \vecnabla\right)\cdot \vec{A}$ is total derivative $\frac{d\veca}{dt}$

            \begin{align}
                \vec{F}= \frac{d\vec{p}}{dt} = -q \left[\frac{d\veca}{dt} + \vec{\nabla}\left(\phi - \vec{v}\cdot\veca\right)\right] \nn
                = - \frac{d}{dt}\left(q\cdot\vec{A}\right)  - q\vecnabla \left(\phi - \vec{v}\cdot \veca    \right) \labtag{I.19} \nonumber
            \end{align}

            \begin{align}
                \rightarrow \frac{d}{dt} \left(\vec{p} + q \cdot \vec{A}\right) = - q \vecnabla \left(\phi - \vec{v}\cdot\veca\right) \labtag{I.20} \nonumber
            \end{align}

            \# 11:48
            Remind classical mechanics: motion of a neutral particle in some potential V:
            \begin{align}
                \frac{d\vec{p}}{dt} =-\vecnabla V \labtag{I.21} \nonumber
            \end{align}

            This yields the formal(!) analogy
            \begin{align}
                \vec{p} \rightarrow \vec{p} + q\cdot \vec{A} =: \vec{p}_{\text{con}} (= \Pi) \nonumber \labtag{I.22a}
            \end{align}
            where $\vec{p}_{\text{con}}$ is called canonical momentum(gauge dependent), and 
            \begin{align}
                V \rightarrow q\cdot\left(\phi - \vec{v}\veca\right) \nonumber \labtag{I.22b}
            \end{align}

            \#11:55-11:58

            and $\vec{p}$ is thus called mechanical momentum(observable). \\
            The mechanical momentum is a physical observable, while the canonical momentum is gauge dependent! and thus can not 
            be an observable!
        




        \subsection{Homogeneous wave Equation}
            In the following we consider electrodynamic fields in uncharged isolators that form a linear homogenous medium, so there are 
            no free charges or currents.

            Maxwell equations in this case:
            \begin{align}
                &\text{(a):} \vec{\nabla}\cdot \vecd = 0 \Rightarrow \vec{\nabla}\vece = 0 \labtag{I.23a} \nn
                &\text{(b):} \vecnabla \cdot \vecb = 0 \labtag{I.23b} \nn
                &\text{(c):}\vecnabla \times \vece = - \partialt \vec{B} \labtag{I.23c}\nn
               \vec{\nabla}\times \vech = \partialt \vecd \rightarrow \frac{1}{\mu}\vecnabla \times\vec{B} =\varepsilon \partialt \vec{E} \Rightarrow   &\text{(d):}  \vec{\nabla}\times \vech = \mu\varepsilon  \partialt \vec{E} \labtag{I.23d} \nonumber
            \end{align}

            The coupled equations \ref{tag: I.23c} and \ref{tag: I.23d} can be decoupled by applying the curl on them:
            
            (c):
            \begin{align}
                \vec{\nabla}\times\left(\vecnabla\times\vece\right) \txtoneq{\ref{tag: I.5} to \ref{tag: I.6}} \vec{\nabla}\cdot \left(\vecnabla\cdot\vece\right) - \vecnabla^2\vec{E} \txtoneq{\ref{tag: I.23c}} \vecnabla \times \left(\partialt \vec{B}\right)\nn
                = - \partialt \left(\vecnabla \times \vecb\right) \nn
                \txtoneq{\ref{tag: I.23d}} - \partialt \left(\mu\varepsilon\partialt \vece\right) \labtag{I.24} \nonumber
            \end{align}

            and
            \begin{align}
                \Delta \vece = \mu\varepsilon \partialtt \vec{E} \labtag{I.25} \nonumber
            \end{align}

            (d):
            \begin{align}
                \vecnabla \times \left(\vecnabla\times\vecb\right) = \dots \labtag{I.26} \nn
                \rightarrow \Delta \vecb = \mu\varepsilon\partialtt \vec{B} \labtag{I.27} \nonumber
            \end{align}

            The result are two decoupled equations(of second order) which can be rewritten using the d'Alembert operator:
            \begin{align}
                \square \psi\left(\vec{r} , t \right) = 0 \labtag{I.28} \nonumber
            \end{align}

            where $\psi = \vec{E}$ or $\psi = \vec{B}$

            Remind, exactly the same type of equations, but with the scalar and the vector potentials instead of the E or the B fields were obtianed as determining equations within the
             Lorenz gauge!!! \#12:30

             This central equation \ref{tag: I.28} is known as homogeneous wave equation.

             The solution manifold of the h.w.e. turns out to be extremely large! In fact every function of the type
             \begin{align}
                 \psi(\vecr,t) =  f_+\left(\veck\cdot \vecr + \omega t\right) + f_-\left(\veck\cdot \vecr + \omega t\right) \labtag{I.29} \nonumber
             \end{align}
             is a solution, if the functions f are differentiable and if the following a certain fixed relation between position and time behavior is fulfilled:
            \begin{align}
                \Delta \psi\left(\vecr,t\right) = \frac{\partial^2 \psi}{\partial x^2} + \frac{\partial^2 \psi}{\partial y^2} + \frac{\partial^2 \psi}{\partial z^2} \labtag{I.30} \nn
                \frac{\partial^2 \psi\left(\vecr,t\right)}{\partial x^2}  \txtoneq{\ref{tag: I.29}} \frac{\partial^2}{\partial x^2} \left[f_+\left(\alpha_+\right) + f_-(\alpha_-)\right], \alpha_\pm = \vec{k}\vec{r} \pm \omega t \nn
                = \frac{\partial}{\partial x} \left[\frac{\partial \alpha_+}{\partial x} \cdot \frac{\partial f_+\left(\alpha_+\right)}{\partial \alpha_+} + \frac{\partial \alpha_-}{\partial x} \cdot \frac{\partial f_-\left(\alpha_-\right)}{\partial \alpha_-}\right] \nn
                = \frac{\partial}{\partial x} \left[k_x \cdot \frac{\partial f_+\left(\alpha_+\right)}{\partial \alpha_+} +k_x \cdot \frac{\partial f_-\left(\alpha_-\right)}{\partial \alpha_-}\right] \nn
                = k_x^2\cdot \left[\frac{\partial^2f_+(\alpha_+)}{\partial_{\alpha_+}^2} + \frac{\partial^2f_-(\alpha_-)}{\partial_{\alpha_-}^2}\right] \labtag{I.31}\nonumber
            \end{align}

            \begin{align}
                \Rightarrow \Delta \psi(\vecr,t) = \vecnabla^2 \psi(\vecr,t) = k^2\cdot \left[\frac{\partial^2f_+(\alpha_+)}{\partial_{\alpha_+}^2} + \frac{\partial^2f_-(\alpha_-)}{\partial_{\alpha_-}^2}\right] \labtag{I.32}\nonumber
            \end{align}

            \# 12:47-12:48










\lecture{4}{20.11.06}{Prof. Alejandro Saenz}{Guangyu He}{Fundamentals of Optical Sciences:theory part}{WS 2020/21}

        Reminder(last lecture):
        \begin{align}
            \square \psi(\vecr,t) = 0 \nonumber
        \end{align}
        where$\psi = \vece,\vecb$ or $\phi,\veca$. Note, for $\vece$ and $\vecb$, if the curl is applied to m.w.e for $\phi$ and $\veca$, if the Lorenz gauge is adopted for m.w.e.

        \# 09:19

        from Eq.\ref{tag: I.29} and Eq.\ref{tag: I.32}  analogously:
        \begin{align}
            \frac{1}{u^2}\partialtt \psi(\vecr,t) \txtoneq{\ref{tag: I.29}} \frac{1}{u^2} \partialtt \left[f_+(\alpha_+) + f_-(\alpha_-)\right] \nn
            = \frac{1}{u^2} \omega^2 \left[\frac{\partial^2 f_+(\alpha_+) }{\partial \alpha_+^2} + \frac{\partial^2 f_-(\alpha_-) }{\partial \alpha_-^2}\right] \labtag{I.33}
        \end{align}

        comparing left and right hand side:
        \begin{align}
            \Delta \psi = k^2 \left[\frac{\partial^2 f_+(\alpha_+) }{\partial \alpha_+^2} + \frac{\partial^2 f_-(\alpha_-) }{\partial \alpha_-^2}\right] \txtoneq{!} \frac{\omega^2}{u^2} \left[\frac{\partial^2 f_+(\alpha_+) }{\partial \alpha_+^2} + \frac{\partial^2 f_-(\alpha_-) }{\partial \alpha_-^2}\right] = \frac{1}{u^2} \partialtt \psi(\vecr,t) \labtag{I.34} \nn
            \Rightarrow \psi(\vecr,t) = f_+(\veck\vecr+\omega t) + f_-(\veck\vecr - \omega t) \nonumber
        \end{align}

        \# 09:29
        ??? Eq.\ref{tag: I.28}, if
        \begin{align}
            k^2 = \frac{w^2}{u^2} \Rightarrow k = \frac{\omega}{u} \text{   or  }\omega = u\cdot k \labtag{I.35}
        \end{align}
        dispersion relation!\# 09:31

        Remind:
        \begin{itemize}
            \item any possible function $f$ as long as it sufficiently differentiable with respect to $t$ and $r$ is a solution, provided the argument fullfil the dispersion relation.
            \item we just solved a differential equation, the solution to a physiocal situation(experimental set-up) follows from considering the corresponding boundary conditions.
            \item it should be emphasized again that the solution of the h.w.e.  is only a solution of the curl applied to the m.w.e determining the curl of $\vece$ or $\vecb$, while it is a solution to the m.w.e for $\phi$ and $\veca$ in Lorenz gauge.
        \end{itemize}

        Specific periodic solutions are, e.g.,
        \begin{align}
            \tilde{f}_\pm (\vecr,t) = \tilde{A}_\pm \cos\left(\veck\vecr \pm \omega t + \phi_0\right) \labtag{I.36a}\nn
            f_\pm(\vecr,t) = A_\pm \cdot e^{i\left(\veck\vecr \pm \omega t + \phi_0\right)} \labtag{I.36b}
        \end{align}
        with a constant $\phi_0$

        the argument $\phi_\pm = \veck\vecr \omega t + \phi_0 $ is called phase.

        For constant time t$t$ the planes of constant phase are defined by $\veck\vecr = \text{const.}$

        In general(for non-constant time t): 
        \begin{align}
            \phi_\pm = \veck\vecr \pm \omega t + \phi_0 =: \alpha_\pm \Leftrightarrow \beta_\pm := \alpha_\pm - \phi_0 = \veck\vecr \pm \omega t \labtag{I.37}
        \end{align}

        Scalar product yields: $\veck\vecr = k \cdot r_k, k=\abs{\vec{k}},r_k = \abs{\vecr_k}$ where $\vecr_k$ projection of $\vecr,\veck$,and
        \begin{align}
            \rightarrow \beta_\pm = kr_k \pm \omega t \nn
            \rightarrow r_k = \frac{\beta_\pm}{k} \mp \frac{\omega}{k}t \labtag{I.38}
        \end{align}

        A plane defined by constant phase $\beta_\pm$ moves with the phase velocity.
        \begin{align}
            \frac{dr_k}{dt} = \mp \frac{\omega}{k} \txtoneq{\ref{tag: I.35}} \mp u \labtag{I.39}
        \end{align}

        The planes move along the wave vector $\veck$. More accurately, $f_-$ moves parallel, $f_+$ moves antiparallel to $\veck$.\\
        The phase velocity $u= \frac{1}{\sqrt{\varepsilon_0\mu_0}}$ says that electromagnetic waves in vacuum move with the vacuum speed of light c.

        In a isotropic linear medium one has $ u = \frac{1}{\sqrt{\varepsilon\mu}}$ instead.

        Usually$\varepsilon > \varepsilon_0$ and $\mu \simeq \mu_0$, the propagation velocity of e.m. in medium is usually slow than in vacuum.

        The ratio $ u := \sqrt{\frac{\varepsilon\mu}{\varepsilon_0\mu_0}}$ is called index of refraction.

        For the here discussed solutions $f$ and $f$ for fixed times of the surfaces of equal phases and thus equal function values repeat periodically in space for
        \begin{align}
            \varphi_1 = \varphi_2 + 2\pi n \text{   with    } n \in N \labtag{I.40}
        \end{align}

        Going along the wave vector $\veck$ the distance between two points of equation function values is
        \begin{align}
            \abs{\vecr_1 - \vecr_2} = \frac{2\pi n}{k} \nonumber
        \end{align}

        The shortest distance(n=1) is called wavelength $\lambda$:
        \begin{align}
            \lambda := \frac{2\pi}{k} \labtag{I.41}
        \end{align}

        If one considers the value of the function atr a fixed point in space then the value repeats periodically after the time period:
        \begin{align}
            T = \frac{2\pi}{\omega} \labtag{I.42}
        \end{align}

        Frequency and angular frequency:
        \begin{align}
            \nu  := \frac{1}{T} \nn
            \omega = 2\pi\nu \nonumber
        \end{align}

        and 
        \begin{align}
            u = \frac{2\pi\nu}{k} = \lambda \nu = \frac{\lambda}{T} \labtag{I.43}
        \end{align}

        As already said, for $\phi$ and $\veca$(in Lorenz gauge!!!) the solutions of the h.w.e. $\psi$ is the solution of the problem! However, in the case of $\vece$ and $\vecb$, we have considered so far only 2 of the four m.w.e! 
        In fact, we solved the equations resolution from applying the  curl onto them(for decoupling $\vece$ and $\vecb$).

        Therefore, we have to look for the (sub)set of solutions that also! solve the remaining m.w.e.!
        \begin{align}
            \vec{E}_\pm = \vec{E}_0 \cdot e^{i(\veck\vecr \pm \omega t + \phi_0)}, \vec{B}_\pm = \vecb_0 \cdot e^{(\veck^\prime\vecr \pm \omega^\prime t + \phi_0^\prime)} \labtag{I.44}
        \end{align}

        In fact, as those were the solutions of the curl applied to the m.w.e (if considering $\vece$ and $\vecb$), we want to have the solutions for the original m.w.e.:
        \begin{align}
            \vecnabla \times \vece = - \partialt \vecb \nonumber
        \end{align}

        Reminder:
        \begin{align}
            \vecnabla \times (f\cdot \vec{g}) = f \cdot \left(\vecnabla \times \vec{g}\right) - \vec{g}\times(\vecnabla f)  \nn
            = f\cdot(\vecnabla \times \vec{g}) + (\vecnabla f)\times \vec{g} \labtag{I.45}
        \end{align}

        left hand sides:
        \begin{align}
            \vec{\nabla}\times \vece_\pm = e^{i (\veck\vecr \pm \omega t + \varphi_0)} (\vecnabla \times\vece_0) - \vece_0 \times \left(\vecnabla e^{i (\veck\vecr \pm \omega t + \varphi_0)}\right)  \nn
            \vec{\nabla}\times \vece_\pm = - \vec{E}_0 \times \left(\vecnabla e^{i (\veck\vecr \pm \omega t + \varphi_0)}\right) \labtag{I.46} \\
            = - i \left(\vec{E}_0 \times \veck \right) \cdot e^{i (\veck\vecr \pm \omega t + \varphi_0)} \nn
            = + i (\veck \times \vece_0) \cdot e^{i (\veck\vecr \pm \omega t + \varphi_0)} \labtag{I.47}
        \end{align}

        right hand sides of \ref{tag: I.23c}:
        \begin{align}
            -\partialt\vecb_\pm \txtoneq{\ref{tag: I.44}} - (\pm i \omega^\prime)e^{i (\veck^\prime\vecr \pm \omega^\prime t + \varphi_0^\prime)} \nn
            = \mp i \omega^\prime \vecb_0 e^{i (\veck^\prime\vecr \pm \omega^\prime t + \varphi_0^\prime)} \labtag{I.48}
        \end{align}

        In order that m.w.e \ref{tag: I.23c} is fulfilled at any instant of time at any place we must require
        \begin{align}
            i(\veck\times\vece_0)e^{i (\veck\vecr \pm \omega t + \varphi_0)} \txtoneq{!} \mp i \omega^\prime \vecb_0 e^{i (\veck^\prime\vecr \pm \omega^\prime t + \varphi_0^\prime)} \labtag{I.49}
        \end{align}
        therefore,
        \begin{align}
            \veck^\prime = \veck, \omega^\prime = \omega, \varphi^\prime_0 = \varphi_0 \labtag{I.50}
        \end{align}

        and
        \begin{align}
            \veck \times \vece_0 \txtoneq{!} \mp \omega \vecb_0 \labtag{I.51}
        \end{align}



        







\lecture{5}{20.11.06}{Prof. Alejandro Saenz}{Guangyu He}{Fundamentals of Optical Sciences:theory part}{WS 2020/21}
see screen cut at \# 11:20-11:22

Convention!
\begin{align}
    \veck \times \vece_0 = + \omega \vecb_0 \labtag{I.58a} \\
    \veck \times \vecb_0 = - \frac{\omega}{u^2} \vece_0 \labtag{I.58b}
\end{align}

This convention leads to the wave-vector being parallel(and not antiparallel) to the propagation direction.

$\vece,\vecb$ and $\veck$ form a right-hand system.

\# 11:25

electromagnetic radiation: transversal waves, i.e. the $\vece$ and $\vecb$ field are orthogonal to the propagation direction.

While physical observable are real and thus only the real solution $\tilde{f}_\pm$ are proper solutions for physical observables like the $\vece$ and the $\vecb$ field, 
it is nevertheless from the mathematical ! point of view often convenient to use the complex solution $f_\pm$ instead.

{\bf The observables is obtained from the complex function f by considering its real part(not its absolute value!)}

Note, while $\vece$ and $\vecb$ are bound to be orthogonal to $\veck$, and to each other, within their plane, $\vece$ and $\vecb$ can be oriented arbitrarily $\Rightarrow$ polarization.

Consider $\veck // \hate_z$
\begin{align}
    \rightarrow E_{0x} = \abs{E_{0x}}\cdot e^{i\phi_0} \text{   and     } E_{0y} = \abs{E_{0y}}\cdot e^{i(\phi_0 + \delta)} \nn
    \rightarrow E_{0x} = \abs{E_{0x}}\cdot \cos(k_z - \omega t + \phi_0) \cdot \hate_x + E_{0x} = \abs{E_{0y}}\cdot \cos(k_z - \omega t + \phi_0 + \delta) \cdot \hate_y \labtag{I.59}
\end{align}

here, $\delta$ is te relative phase between the x and the y components of $\vece$.

Thus in the complex representation of electromagnetic fields, this relative phase $\delta$ defines the polarization of the field.

(I). $\delta = 0$ pr $\pm \pi$: linear polarization
\begin{align}
    \vece = \left(\abs{E_{0x}}\hate_x \pm \abs{E_{0y}}\hate_y\right) \cdot \cos(kz - \omega t + \phi_0) \labtag{I.60}\\
    \abs{\vece} = \sqrt{\abs{E_{0x}}^2 + \abs{E_{0y}}^2} \labtag{I.61}
\end{align}

remaind: $\abs{E_{0x}}$ and $\abs{E_{0y}}$are independent on $\vecr$ and $t$ $\rightarrow$ constant. see screen shot at \# 11:56

While the direction of the $\vece$ field is in this case fixed(within the x,y plane), its values is modulated by the cos(...) and thus varies periodically from -1 to +1(in space and time).

(II). $\delta = \pm \frac{\pi}{2}$ and $\abs{E_{0x}} = \abs{E_{0y}} =: E_0$: circular polarization
\begin{align}
    \vece = \vece_0 \left[\cos(kz - \omega t + \phi_0)\hate_x \mp \sin(kz-\omega t +\phi_0)\hate_y\right] \labtag{I.62}
\end{align}
Seen at a fixed point in space, eq\ref{tag: I.26} corresponds to the parametrization of a circle, and the $\vece$ field vector rotates with $\omega$ in either positive or negative direction. screenshot at \# 12:07

(III). $\delta = \pm \frac{\pi}{2}, \abs{E_{0x}} \neq \abs{E_{0y}}$: elliptical polarization

(IV). 

Note: it is really important that you get a feeling for polarizations "playing" around with them, for example using a program like EMANIM.

There is different ways to specify the polarization, since in the end of the day one has to fix only two components of a vector in the plane orthogonal to the propagation direction. 
In the complex representation this was done using $\abs{E_{0x}}$ and $\abs{E_{0y}}$.\\
One other representation for the polarization only is the one via Jones matrices.

In this case there presentation is done via a normalized! 2-d vector see \# 12:28-33

Note: for linear polarization the Jones vectors are real, fo elliptical or circular polarization they are complex.

Example fpr tje usefulness of the Jones matrices: consider the superposition of a left and a right circular polarized wave:
\begin{align}
    \frac{1}{\sqrt{2}} \left(\begin{matrix}
        1 \\
        i
    \end{matrix}\right) + \frac{1}{\sqrt{2}} \left(\begin{matrix}
        1 \\
        -i
    \end{matrix}\right) = \sqrt{2} \left(\begin{matrix}
        1 \\
        0
    \end{matrix}\right) \rightarrow \left(\begin{matrix}
        1 \\
        0
    \end{matrix}\right) \nonumber
\end{align}

Consider now the general form of electromagnetic waves that are solutions of the m.w.e(but remind that the full solutions depend on the boundary conditions!):
\begin{align}
    \vece(\vecr,t) = \vece_0\cdot e^{i(\veck\vecr-\omega t + \phi_0)} = \vece_0 \cdot e^{i(\veck\vecr-\omega t + \phi_0) }\cdot\hate_E \labtag{I.66a} \\
    \vecb(\vecr,t) = \vecb_0\cdot e^{i(\veck\vecr-\omega t + \phi_0)} = \frac{E_0}{u} \cdot e^{i(\veck\vecr-\omega t + \phi_0)}\cdot(\hate_k\times\hate_E) \labtag{I.66b}
\end{align}

energy density $w$ of the em field represented by eq\ref{tag: I.66a}and \ref{I.66b}
\begin{align}
    w_{\text{EB}} = \frac{1}{2}\left[(\vece\cdot\vecd) + (\vech\cdot\vecb)\right] \labtag{I.67}
\end{align}
in a linear homogeneous medium
\begin{align}
    w_{\text{EB}} = \frac{1}{2}\left[\varepsilon\vece^2 + \frac{1}{\mu\vecb^2}\right] \labtag{I.68}
\end{align}


\end{document}