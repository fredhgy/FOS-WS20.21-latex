%last updated 01.11.19 by GuangyuHe

\documentclass[twoside]{article}
%\documentclass[UTF8]{ctexart}
\setlength{\oddsidemargin}{0.25 in}
\setlength{\evensidemargin}{-0.25 in}
\setlength{\topmargin}{-0.6 in}
\setlength{\textwidth}{6.5 in}
\setlength{\textheight}{8.5 in}
\setlength{\headsep}{0.75 in}
\setlength{\parindent}{0 in}
\setlength{\parskip}{0.1 in}

%
% ADD PACKAGES here:
%

\usepackage[UTF8]{ctex}

\usepackage{amsmath,amsfonts,graphicx}
\usepackage{esint}
\usepackage{multicol}
%\usepackage{MnSymbol}
\usepackage{mathtools}
\usepackage[bottom]{footmisc}
\usepackage[pdfpagelabels,hyperindex,hyperfigures,breaklinks,colorlinks,linkcolor=black,citecolor=black,urlcolor=black]{hyperref}
\usepackage{geometry}
\usepackage{marginnote}
\usepackage{color}
%
% The following commands set up the lecnum (lecture number)
% counter and make various numbering schemes work relative
% to the lecture number.
%

\newcounter{lecnum}
\renewcommand{\thepage}{\thelecnum-\arabic{page}}
%\renewcommand{\thesection}{\thelecnum.\arabic{section}}
%\renewcommand{\theequation}{\thelecnum.\arabic{equation}}
%\renewcommand{\thefigure}{\thelecnum.\arabic{figure}}
%\renewcommand{\thetable}{\thelecnum.\arabic{table}}

%
% The following macro is used to generate the header.
%
\newcommand{\lecture}[6]{
   \pagestyle{myheadings}
   \thispagestyle{plain}
   \newpage
   \setcounter{lecnum}{#1}
   \setcounter{page}{1}
   \noindent
   \begin{center}
   \framebox{
      \vbox{\vspace{2mm}
    \hbox to 6.28in { {\bf #5
	\hfill #6} }
       \vspace{4mm}
       \hbox to 6.28in { {\Large \hfill Lecture #1: #2  \hfill} }
       \vspace{2mm}
       \hbox to 6.28in { {\it Lecturer: #3 \hfill Scribes: #4} }
      \vspace{2mm}}
   }
   \end{center}
   \markboth{Lecture #1: #2}{Lecture #1: #2}

   %{\bf Note}: {\it LaTeX template courtesy of UC Berkeley EECS dept.}

   %{\bf Disclaimer}: {\it These notes have not been subjected to the
   %usual scrutiny reserved for formal publications.  They may be distributed
  % outside this class only with the permission of the Instructor.}
   \vspace*{4mm}
}

\newcommand{\exercise}[6]{
   \pagestyle{myheadings}
   \thispagestyle{plain}
   \newpage
   \setcounter{lecnum}{#1}
   \setcounter{page}{1}
   \noindent
   \begin{center}
   \framebox{
      \vbox{\vspace{2mm}
    \hbox to 6.28in { {\bf #5
	\hfill #6} }
       \vspace{4mm}
       \hbox to 6.28in { {\Large \hfill Exercise #1: #2  \hfill} }
       \vspace{2mm}
       \hbox to 6.28in { {\it Lecturer: #3 \hfill Scribes: #4} }
      \vspace{2mm}}
   }
   \end{center}
   \markboth{Exercise #1: #2}{Exercise #1: #2}
   
   %{\bf Note}: {\it LaTeX template courtesy of UC Berkeley EECS dept.}

   %{\bf Disclaimer}: {\it These notes have not been subjected to the
   %usual scrutiny reserved for formal publications.  They may be distributed
  % outside this class only with the permission of the Instructor.}
  \vspace*{4mm}
}

\newcommand{\review}[4]{
   \pagestyle{myheadings}
   \thispagestyle{plain}
   \newpage
   %\setcounter{lecnum}{#1}
   \setcounter{page}{1}
   \noindent
   \begin{center}
   \framebox{
      \vbox{\vspace{2mm}
    \hbox to 6.28in { {\bf #3
	\hfill #4} }
       \vspace{4mm}
       \hbox to 6.28in { {\Large \hfill Review \hfill} }
       \vspace{2mm}
       \hbox to 6.28in { {\it Lecturer: #1 \hfill Scribes: #2} }
      \vspace{2mm}}
   }
   \end{center}
   \markboth{Review}{Review}
   
   %{\bf Note}: {\it LaTeX template courtesy of UC Berkeley EECS dept.}

   %{\bf Disclaimer}: {\it These notes have not been subjected to the
   %usual scrutiny reserved for formal publications.  They may be distributed
  % outside this class only with the permission of the Instructor.}
  \vspace*{4mm}
}

\newcommand{\answer}[6]{
   \pagestyle{myheadings}
   \thispagestyle{plain}
   \newpage
   \setcounter{lecnum}{#1}
   \setcounter{page}{1}
   \noindent
   \begin{center}
   \framebox{
      \vbox{\vspace{2mm}
    \hbox to 6.28in { {\bf #5
	\hfill #6} }
       \vspace{4mm}
       \hbox to 6.28in { {\Large \hfill Answer Sheet #1: #2  \hfill} }
       \vspace{2mm}
       \hbox to 6.28in { {\hfill #3 , #4} }
      \vspace{2mm}}
   }
   \end{center}
   \markboth{Answer Sheet #1: #2}{Answer Sheet #1: #2}
   
   %{\bf Note}: {\it LaTeX template courtesy of UC Berkeley EECS dept.}

   %{\bf Disclaimer}: {\it These notes have not been subjected to the
   %usual scrutiny reserved for formal publications.  They may be distributed
  % outside this class only with the permission of the Instructor.}
  \vspace*{4mm}
}


%
% Convention for citations is authors' initials followed by the year.
% For example, to cite a paper by Leighton and Maggs you would type
% \cite{LM89}, and to cite a paper by Strassen you would type \cite{S69}.
% (To avoid bibliography problems, for now we redefine the \cite command.)
% Also commands that create a suitable format for the reference list.
\renewcommand{\cite}[1]{[#1]}
\def\beginrefs{\begin{list}%
        {[\arabic{equation}]}{\usecounter{equation}
         \setlength{\leftmargin}{2.0truecm}\setlength{\labelsep}{0.4truecm}%
         \setlength{\labelwidth}{1.6truecm}}}
\def\endrefs{\end{list}}
\def\bibentry#1{\item[\hbox{[#1]}]}

%Use this command for a figure; it puts a figure in wherever you want it.
%usage: \fig{NUMBER}{CAPTION}{picture scale}{picture name}
\newcommand{\fig}[4]{
    \begin{center}
        \includegraphics[scale=#1]{fig//#2}
    \end{center}
    \begin{center}
        Figure #3 #4
    \end{center}


    %\begin{center}
    %    \includegraphics[scale=#3]{#4}
    %\end{center}
    %\vspace{2mm}
    %\begin{center}            
    %    Figure \thelecnum.#1:~#2
    %\end{center}
}
% Use these for theorems, lemmas, proofs, etc.
%\newtheorem{theorem}{Theorem}[lecnum]
%\newtheorem{lemma}[theorem]{Lemma}
%\newtheorem{proposition}[theorem]{Proposition}
%\newtheorem{claim}[theorem]{Claim}
%\newtheorem{corollary}[theorem]{Corollary}
%\newtheorem{definition}[theorem]{Definition}
%\newenvironment{proof}{{\bf Proof:}}{\hfill\rule{2mm}{2mm}}

% **** IF YOU WANT TO DEFINE ADDITIONAL MACROS FOR YOURSELF, PUT THEM HERE:

\newcommand\E{\mathbb{E}}

\newcommand{\nn}{
    \nonumber \\
}%不带公式编号换行

\newcommand{\ketn}[1]{
    | #1 \rangle
}%右矢

\newcommand{\bran}[1]{
    \langle #1 |
}%左矢

\newcommand{\braoket}[1]{
    \langle #1 | #1 \rangle
}

\newcommand{\bratket}[2]{
    \langle #1 | #2 \rangle
}

\newcommand{\inpro}[2]{
    \langle #1 , #2 \rangle
}%内积

\newcommand{\rightarrowtxt}[1]{
    \stackrel{\text{#1}}{\longrightarrow}
}%箭头上带文字

\newcommand{\ahat}{
    \hat{a}
}%下降算符

\newcommand{\adag}{
    \hat{a}^\dagger
}%上升算符

\newcommand{\osrt}{
    \frac{1}{\sqrt{2}}
}%根二分之一

\newcommand{\abs}[1]{
    \left| #1 \right|
}%绝对值

\newcommand{\sumint}{
    \mathclap{\displaystyle\int}\mathclap{\textstyle\sum}
}%叠加积分

\newcommand{\longline}{
    \rule{\textwidth}{0.5mm}
} %页面分割线

\newcommand{\explain}[1]{
    \footnote{{\it #1}}
}%脚注+斜体说明

\newcommand{\sidemark}[1]{
    \reversemarginpar
    \marginnote{\textsl{side remark:} \\ #1 }[3cm]
}%侧边注

\newcommand{\highlightquestion}[1]{
    {\color{red} #1 }
}%红色高亮问题

\newcommand{\highlight}[1]{
    {\color{blue} #1 }
}%蓝色高亮重点

\newcommand{\sbeq}{
    \overset{!}{=}
}%should be equal

\newcommand{\bigkuohao}[2]{
    \left\{
    \begin{aligned}
   #1 \\
    #2
    \end{aligned}
    \right.
}

\newcommand{\txtoneq}[1]{
    \stackrel{\text{#1}}{=}
}%等号上带文字

\newcommand{\hate}{
    \hat{e}
}

\newcommand{\veca}{
    \vec{A}
}

\newcommand{\vecb}{
    \vec{B}
}

\newcommand{\vecd}{
    \vec{D}
}

\newcommand{\vece}{
    \vec{E}
}

\newcommand{\vech}{
    \vec{H}
}

\newcommand{\vecj}{
    \vec{J}
}

\newcommand{\vecjj}{
    \vec{j}
}

\newcommand{\veck}{
    \vec{k}
}

\newcommand{\vecp}{
    \vec{P}
}

\newcommand{\vecpp}{
    \vec{p}
}

\newcommand{\vecs}{
    \vec{S}
}

\newcommand{\vecr}{
    \vec{r}
}

\newcommand{\vecnabla}{
    \vec{\nabla}
}

\newcommand{\partialt}{
    \frac{\partial}{\partial t}
}

\newcommand{\partialtt}{
    \frac{\partial^2}{\partial t^2}
}

\newcommand{\labtag}[1]{
    \label{tag: #1}\tag{#1}
}







%command for physics uses


%\begin{document}
    % Some general latex examples and examples making use of the
% macros follow.  
%**** IN GENERAL, BE BRIEF. LONG SCRIBE NOTES, NO MATTER HOW WELL WRITTEN,
%**** ARE NEVER READ BY ANYBODY.

%\section{Some theorems and stuff} % Don't be this informal in your notes!

%\begin{lemma} %引理
%This is the first lemma of the lecture.
%\end{lemma}

%\begin{proof} %证明
%The proof is by induction on $\ldots$.
%For fun, we throw in a figure.
%%%NOTE USAGE !
%\fig{1}{1in}{A Fun Figure} %插入图片说明

%This is the end of the proof, which is marked with a little box.
%\end{proof}

%\subsection{A few items of note}

%Here is an itemized list:
%\begin{itemize}
%\item this is the first item;
%\item this is the second item.
%\end{itemize}

%Here is an enumerated list:
%\begin{enumerate}
%\item this is the first item;
%\item this is the second item.
%\end{enumerate}

%Here is an exercise:

%{\bf Exercise:}  Show that ${\rm P}\ne{\rm NP}$.

%Here is how to define things in the proper mathematical style.
%Let $f_k$ be the $AND-OR$ function, defined by

%\[ f_k(x_1, x_2, \ldots, x_{2^k}) = \left\{ \begin{array}{ll}

%	x_1 & \mbox{if $k = 0$;} \\

%	AND(f_{k-1}(x_1, \ldots, x_{2^{k-1}}),
%	   f_{k-1}(x_{2^{k-1} + 1}, \ldots, x_{2^k}))
%	 & \mbox{if $k$ is even;} \\

%	OR(f_{k-1}(x_1, \ldots, x_{2^{k-1}}),
%	   f_{k-1}(x_{2^{k-1} + 1}, \ldots, x_{2^k}))	
%	& \mbox{otherwise.} 
%	\end{array}
%	\right. \]

%\begin{theorem}
%This is the first theorem.
%\end{theorem}

%\begin{proof}
%This is the proof of the first theorem. We show how to write pseudo-code now.
%*** USE PSEUDO-CODE ONLY IF IT IS CLEARER THAN AN ENGLISH DESCRIPTION

%Consider a comparison between $x$ and~$y$:
%\begin{tabbing}
%\hspace*{.25in} \= \hspace*{.25in} \= \hspace*{.25in} \= \hspace*{.25in} \= \hspace*{.25in} \=\kill
%\>{\bf if} $x$ or $y$ or both are in $S$ {\bf then } \\
%\>\> answer accordingly \\
%\>{\bf else} \\
%\>\>    Make the element with the larger score (say $x$) win the comparison \\
%\>\> {\bf if} $F(x) + F(y) < \frac{n}{t-1}$ {\bf then} \\%
%\>\>\> $F(x) \leftarrow F(x) + F(y)$ \\
%\>\>\> $F(y) \leftarrow 0$ \\
%\>\> {\bf else}  \\
%\>\>\> $S \leftarrow S \cup \{ x \} $ \\
%\>\>\> $r \leftarrow r+1$ \\
%\>\> {\bf endif} \\
%\>{\bf endif} 
%\end{tabbing}

%This concludes the proof.
%\end{proof}


%\section{Next topic}

%Here is a citation, just for fun~\cite{CW87}.

%\section*{References}
%\beginrefs
%\bibentry{CW87}{\sc D.~Coppersmith} and {\sc S.~Winograd}, 
%``Matrix multiplication via arithmetic progressions,''
%{\it Proceedings of the 19th ACM Symposium on Theory of Computing},
%1987, pp.~1--6.
%\endrefs

% **** THIS ENDS THE EXAMPLES. DON'T DELETE THE FOLLOWING LINE:
%\end{document}
\begin{document}
    \lecture{1}{20.11.04}{Prof. Alejandro Saenz}{Guangyu He}{Fundamentals of Optical Sciences:theory part}{WS 2020/21}

    \setcounter{section}{-1}
    \section{General remarks}
        Exercises: 50\% handed via Moodle in hand-written PDF

        (most-likely) Oral(digital) exam

        Communication via Moodle make sure you check your e-mail account to which Moodle sends e-mails!

        \underline{Content(theory part):}

        I. Basic Electrodynamics,

        II. Light-atom interaction in the semi-classical picture(description),

        III. Quantization of electromagnetic fields,
        \# 09:53-09:57

    \section{Basic Electrodynamics}
    \setcounter{subsection}{-1}
        \subsection{Maxwell equations}

            $$
            \begin{aligned}
                \vecnabla \cdot \vec{D} &= \rho_f \nn
                \vecnabla \cdot \vec{B} &= 0 \nn
                \vecnabla \times \vec{E} &= - \partialt \vec{B} \nn
                \vecnabla \times \vec{H} &= \partialt \vec{D} + \vec{j}_f \nn
            \end{aligned}
            $$

            and:
            $$
            \begin{aligned}
                \vec{D} = \varepsilon \vec{E} + \vec{P} \nn
                \vec{H} = \frac{1}{\mu} \vec{B} + \vec{M} \nn
            \end{aligned}
            $$

            for $\vec{P} = 0$:
            $$
            \vec{D} = \varepsilon \vec{E}
            $$

            where $\varepsilon = \varepsilon_0\left(1+\chi_e\right)$

            alterative: $\vec{D} = \varepsilon_0 \cdot \varepsilon_p \cdot \vec{E} $

            {\sl Don't mess up source!}

            \longline

            Lorentz force law: \# 10:15-10:20
            $$
            \vec{F} = q \left(\vec{E} + \vec{v} \times \vec{B}\right)
            $$
            {\sl Setting charge particals and back field}

        \subsection{Potential picture and gauge transformations}
            \# 10:21

            While due to Faraday's law the electric field can not be written as the gradient of a (scalar) potential as in
            electrostatics, the magnetic field can still be written as the curl of a vector potential:

            \begin{align}
                \vec{B} = \vec{\nabla} \times \vec{A}  \labtag{I.1} \nonumber
            \end{align}

            Insertion of \ref{tag: I.1} into Faraday's law yields
            \begin{align}
                \vecnabla \times \vec{E} = - \partialt{B} \txtoneq{(I.1)} - \partialt\left(\vec{\nabla}\times \vec{A}\right) \nonumber \\
                \rightarrow   \vecnabla \times \vec{E} + \vecnabla \times \left(\partialt \vec{A}\right) = 0 \nonumber \\
                \rightarrow \vecnabla \times \left(\vec{E} + \partialt \vec{A}\right) = 0 \labtag{I.2}
            \end{align}

            \# 10:30

            we thus obtain a new quatitiy with a vanishing curl which thus can be written as the 
            gradient of a scalar potential:
            \begin{align}
                \rightarrow \vec{E} + \partialt\vec{A} =: - \vec{\nabla}\phi \labtag{I.3} \nonumber
            \end{align}
            (note, the choice of the negative sign is only convention)

            This yields in turn for the electric fields:
            \begin{align}
                \vec{E} = - \left(\vecnabla \phi + \partialt \vec{A} \right) \labtag{I.4}
            \end{align}

            Note, when using the scalar potential$\phi$ and the vector potential $\vec{A}$, two of 
            the Maxwell equations are automatically fulfilled, as they were used in the deviations of 
            the potentials. $\left(\vec{\nabla}\cdot\vec{B} = 0, \vecnabla \times \vec{E} = -\partialt \vec{B}\right)$

            \longline

            Potential picture / description: using $\phi$ and $\vec{A}$

            Field picture / description: using $ \vec{E} $ and $\vec{B}$

            Maxwell equations in potential picture:
            \begin{align}
                \text{(i) Gauss:    } \vecnabla \cdot \vec{E} = \frac{1}{\varepsilon_0} \rho \nn
                \rightarrow - \vec{\nabla} \left(\vec{\nabla}\phi + \partialt \vec{A} \right) = \frac{1}{\varepsilon_0} \rho \nn
                \vecnabla^2\phi + \partialt\left(\vec{\nabla}\cdot\vec{A}\right) =  -  \frac{1}{\varepsilon_0} \rho  \labtag{I.5}
            \end{align}










            \lecture{2}{20.11.05}{Prof. Alejandro Saenz}{Guangyu He}{Fundamentals of Optical Sciences:theory part}{WS 2020/21}

            \begin{align}
                \text{(ii) Ampere: }  \vecnabla \times \vec{B} &= \mu_0 \vec{j} + \mu_0 \varepsilon_0 \partialt \vec{E} \nn
                \vec{\nabla} \times \left(\vec{\nabla} \times \vec{A}\right)  &= \mu_0 \vec{j} + \mu_0 \varepsilon_0 \left(\vec{\nabla} \phi + \partialt \vec{A}\right) \nn
                &=  \mu_0 \vec{j} + \mu_0 \varepsilon_0  \left(\vec{\nabla} \partialt \phi + \partialtt \vec{A}\right) \nonumber
            \end{align}
            
            Remind: $ \vecnabla \times \left(\vecnabla \times \vec{f}\right) = \vec{\nabla} \left(\vecnabla \cdot \vec{f}\right) - \nabla^2 \vec{f}$

            \begin{align}
                \rightarrow \vec{\nabla}\cdot \left(\vec{\nabla}\cdot \vec{A}\right) - \nabla^2 \vec{A} = \mu_0 \vec{j} - \mu_0\varepsilon_0\left(\vec{\nabla}\partialt\phi + \partialtt \vec{A}\right) \nn
                \left(\nabla^2 \vec{A} - \mu_0\varepsilon_0 \partialtt\vec{A}\right) - \vec{\nabla}\left(\vec{\nabla}\cdot \vec{A} + \mu_0\varepsilon_0 \partialt \phi\right) = -\mu_0 \vec{j} \labtag{I.6}
            \end{align}

            Besides the fact that the equations determining the potentials do not look very nice, the problem of finding 6 field components has reduced to 4 potential components.

            $E_x,E_y,E_z,B_x,B_y,B_z \longrightarrow A_x,A_y,A_z,\phi$

            {\sl field can be measured and unique, but potential has a kind of freedom.}

            Already from electro- and magnetostatics it is , however, known that the definition of the potentials for give E and B fields is not unique!

            Which freedom in choosing the potentials do we have in electrodynamics:

            \begin{align}
                \vec{A}^\prime = \vec{A} + \vec{a} \nn
                \phi^\prime = \phi + f \nonumber
            \end{align}    

            using \ref{tag: I.1}, i.e. $\vec{B} = \vec{\nabla}\times\vec{A}$,we found
            $$
            \vecnabla \times \vec{A}^\prime \txtoneq{!} \vec{\nabla}\times\vec{A} \Rightarrow \vec{\nabla}\times\vec{a} = 0 \Rightarrow \vec{a} = \vec{\nabla}\phi 
            $$

            using \ref{tag: I.4}, i.e. $\vec{E} = - \left(\vec{\nabla}\phi + \partialt \vec{A}\right)$
            \begin{align}
                \vec{\nabla}\phi^\prime + \partialt \vec{A}^\prime &\txtoneq{!} \vec{\nabla}\phi + \partialt\vec{A}\nn
                \vec{\nabla}\phi + \vec{\nabla}f + \partialt\vec{A} + \partialt\vec{a} &\txtoneq{!} \vec{\nabla} \partialt\vec{A}\nn
                \vecnabla f + \partialt\vec{a} = 0 \Rightarrow \vec{\nabla}f + \partialt\left(\vec{\nabla}\phi\right) &= 0\nn
                \Rightarrow \vec{\nabla} \left(f + \partialt \phi\right) &= 0\nn
                \Rightarrow f &= - \partialt \phi \labtag{I.7}
            \end{align}
            \# 09:43 constant zero

            in fact, the result remian unchanged, if we have $f^\prime = f + c(t)$
            \# 09:45

            and gauge freedom: \# 09:47
            \begin{align}
                \vec{A}^\prime = \vec{A} + \vec{\nabla}\phi \nn
                \phi^\prime = \phi - \partialt \phi \labtag{I.8}
            \end{align}
            
            In other words, the two vector potentials $\vec{A}$ and$\vec{A}^\prime$ yields the same E and B field, if the scalar potential $\phi$ is changed accordingly, and vice versa.

            Equation \ref{tag: I.8} defines thus so-called gauge transformations, the freedom to choose the gauge potential $\phi$ is known as gauge invariance(or gauge freedom).
            
            \# 09:52-09:57

            Where there is in principle an infinite number of possible gauge transformations, tweo special choices for the gauge potential have proven most popular.

            \underline{(I)Coulomb gauge}

            Choice of the potentials such that
            $$
            \vec{\nabla}\cdot \vec{A} \txtoneq{!} 0
            $$
            with this choice one obtains
            \begin{align}
                \vec{\nabla}^2 \phi + \partialt\left(\vecnabla \cdot \vec{A}\right) \txtoneq{\ref{tag: I.5}}  - \frac{\rho}{\varepsilon_0}  \Rightarrow \vec{\nabla}^2 \phi =  - \frac{\rho}{\varepsilon_0} \labtag{I.9} \nonumber
            \end{align}

            which is identical to the Poisson equation know from electrostatics.

            However, for the vector potential, only one term in Equation \ref{tag: I.6} vanishes:
            \begin{align}
                \left(\vec{\nabla}^2\vec{A} - \mu_0 \varepsilon_0 \partialtt \vec{A}\right) = -\mu_0 \vec{j} + \mu_0\varepsilon_0 \vec{\nabla}\left(\partialt \phi\right) \nonumber
            \end{align}

            introducing the d'Alembert operator \# 10:09
            \begin{align}
                \square := \vec{\nabla}^2 - \mu_0 \varepsilon_0 \partialtt = \Delta - \frac{1}{c^2}\partialtt \nonumber
            \end{align}

            we obtain
            \begin{align}
                \square \vec{A}=-\mu_0\vec{j} + \mu_0\varepsilon_0\vec{\nabla}\left(\partialt \phi\right) \labtag{I.10} \nonumber
            \end{align}
            
            \# 10:13
            {\sl using only in vacuum}

            Note: in fact the d'Alembert operator is, in practice not always uniquely defined the way given above, but in a linear, homogeneous
            medium one implies $\varepsilon\mu$ instead of $\varepsilon_0\mu_0$.

            (In fact it can be shown that the vector potential in Coulomb gauge depends solely on the transversal component of the current.)

            \underline{(II) Lorenz gauge}
            \begin{align}
                \vec{\nabla}\cdot \vec{A} \txtoneq{!} -\mu_0 \varepsilon_0 \partialt \phi \nonumber
            \end{align}
            From Eq.\ref{tag: I.5} one gets:
            \begin{align}
                \vec{\nabla}^2 \phi + \partialt\left(\vec{\nabla}\cdot \vec{A}\right) = -\frac{\rho}{\varepsilon_0} \nn
                \Rightarrow \vec{\nabla}^2 \phi - \mu_0\varepsilon_0 \partialtt\phi = -\frac{\rho}{\varepsilon_0} \nn
                \rightarrow \square\phi = -\frac{\rho}{\varepsilon_0} \labtag{I.11} \nonumber
            \end{align}

            and with Eq.\ref{tag: I.6}:
            \begin{align}
                \left(\vec{\nabla}^2\vec{A} -  \mu_0\varepsilon_0 \partialtt\vec{A}\right) - \vec{\nabla}\left(\vec{\nabla}\cdot \vec{A} +  \mu_0\varepsilon_0 \partialt\phi\right) = - \mu_0 \vec{j} \nn
                \rightarrow \square \vec{A} = -\mu_0 \vec{j} \labtag{I.12} \nonumber
            \end{align}
            \# 10:35-10:38 different between two gauge

            Note: in Lorenz gauge the potentials $\vec{A}$ and $\phi$ are completely decoupled and one depends only on the charge density,
            the other only on the current.\\
            Furthermore, note the symmetry of the determining equations for the scalar and the vector potentials.

            While for computations, either gauge may be more efficient, "easier", ..., the Lorenz gauge is typically more appropriate for a deeper understanding of electrodynamics and relativity.
            
            Important: in fact the Lorenz gauge condition does not define the scalar and vector potential uniquely! Thus the Lorenz gauge is actually a gauge class!

\end{document}